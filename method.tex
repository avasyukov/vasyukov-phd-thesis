\section{Численный метод}

Перед тем, как перейти к исследованию полной задачи \ref{matrix_equation} в трёхмерной постановке, рассмотрим одномерное уравнение вида
\begin{equation}
\frac{\partial\vec{u}}{\partial{t}}+\mathbf{A}\frac{\partial\vec{u}}{\partial{x}}=0.
\label{advection_equation}
\end{equation}

\subsection{Решение одномерной задачи}

\subsubsection{Гиперболические свойства определяющей системы уравнений}

Если матрица $\mathbf{A}$ имеет полный набор вещественных собственных значений, 
то такое уравнение называется гиперболическим, и его решения соответствуют 
процессам, которые носят волновой характер. Спектральное исследование матриц $\mathbf{A}_x$, $\mathbf{A}_y$, $\mathbf{A}_z$ проведено в \cite{chelnokov}, где показано, что для них существует полный набор собственных значений и собственных векторов.

В этом случае для любой из матриц $\mathbf{A}_x$, $\mathbf{A}_y$, $\mathbf{A}_z$ существует разложение:
$$\mathbf{A}=\mathbf\Omega^{-1}\mathbf\Lambda\mathbf\Omega,$$
где $\mathbf\Omega$ -- матрица, строки которой $\vec\omega_i^T$ являются собственными для матрицы $\mathbf A$ и
удовлетворяют соотношениям
$$\vec\omega_i^T\mathbf A=\lambda_i\vec\omega_i^T$$ или, что то же самое, транспонированные строки $\mathbf\Omega$ являются собственными векторами для матрицы $\mathbf A^T$
$$\mathbf A^T\vec\omega_i=\lambda_i\vec\omega_i.$$
Здесь $\mathbf\Lambda=diag\{\lambda_i\}$ -- диагональная матрица соответствующих собственных значений.

Домножив уравнение \ref{advection_equation} слева на $\mathbf\Omega$, получаем уравнение
$$\frac{\partial{\mathbf\Omega{\vec u}}}{\partial t}+
\Lambda\frac{\partial{\mathbf\Omega{\vec u}}}{\partial x}=0,$$
которое после перехода к инвариантам Римана ${\vec v}=\mathbf\Omega{\vec u}$ приобретает вид
$$\frac{\partial{\vec v}}{\partial t}+
\Lambda\frac{\partial{\vec v}}{\partial x}=0$$
и тем самым распадается на $n$ одномерных уравнений вида
\begin{equation}
\frac{\partial{v_i}}{\partial t}+\lambda_i\frac{\partial{v_i}}{\partial x}=0.
\label{advection_equation_splitted}
\end{equation}
Таким образом, решение уравнения \ref{advection_equation} представляется в виде
суммы плоских волн, движущихся со скоростями $\lambda_i$.

\subsubsection{Сеточно-характеристический метод}

\todo{Внятный рассказ на пальцах}

- У нас получились независимые уравнения переноса
- Смотрим на каждое из них отдельно и исследуем
- Получаем перенос инвариантов Римана вдоль характеристик
- $\lambda_i$ разные для разных видов матриц $\mathbf A$
- В итоге $\lambda_i$ определяют, из каких точек надо переносить инварианты Римана
- Как получить значения в нужных точках - см.далее
- После того, как все инварианты Римана на $n+1$ слое получены, получаем из них исходный $\vec u$

\todo{Внятно про структурные и бесструктурные сетки. Для структурных - 1ый и 2ой порядок, гибридизация. Для бесструктурных - аналоги.}
\todo{Внятно про курантовский шаг в случае сеток обоих типов.}

\subsubsection{Восстановление значения на предыдущем временном слое}

\todo{Про интерполяцию}

\subsubsection{Расчёт внутренних узлов}
Рассмотрим несколько подробнее метод решения одномерной задачи
\begin{equation}
\frac{\partial}{\partial t}v+\lambda \frac{\partial}{\partial x}v = 0.
\label{one_dim_eq}
\end{equation}
Для её решения предлагается использовать сеточно-характеристический метод, суть
которого состоит в следующем. Из того узла $m$ временного слоя $n+1$, в котором
требуется получить решение, опускаются характеристики.
\begin{figure}[h]
\center{\includegraphics[width=0.5\textwidth]{eps/gcm-idea.eps}}
\caption{Принципиальная схема сеточно-характеристического метода.}
\end{figure}
Из точки пересечения характеристики со слоем $n$ значение $v$ переносится в 
точку $\xi^{n+1}_m$:
$$v_i^{n+1}(\xi_m)=v^{n}_i(\xi_m-\lambda_i\tau).$$
Если характеристика не попадает точно в расчётный узел, то применяются различные
методы реконструкции значения в данной точке (в данной работе используется
интерполяция).

\subsubsection{Расчёт граничных узлов}

\todo{Внятно написать эти два раздела}

\subsubsection{Расчёт контактных узлов}

Метод, описанный в предыдущем пункте, годится лишь для расчёта внутренних узлов
сетки, т.е. только в том случае, если характеристика, выпущенная из узла, не
выводит за пределы области интегрирования. В случае, когда узел является
внешним, применяется иной подход для решения задачи. Рассматриваемая система
уравнений в граничных узлах области интегрирования имеет не больше трёх
\cite{chelnokov} выводящих характеристик, поэтому для корректной постановки
задачи требуется задание граничных условий для каждого внешнего узла сетки в
количестве, равном числу выводящих характеристик. Граничные условия могут быть
нескольких видов (символы без волны -- для первого тела, с волной -- для второго):
\begin{itemize}
\item{свободная граница
\begin{eqnarray}
\sigma_\tau=\sigma_n=0; \nonumber
\end{eqnarray}}
\item{скольжение тел друг относительно друга 
\begin{eqnarray}
v_n=\tilde{v}_n,\nonumber\\
\sigma_n=\tilde{\sigma}_n,\nonumber\\
\sigma_\tau=\tilde{\sigma}_\tau=0; \nonumber
\end{eqnarray}}
\item{слипание тел
\begin{eqnarray}
v_n=\tilde{v}_n,\nonumber\\
v_\tau=\tilde{v}_\tau.
\end{eqnarray}}
\end{itemize}
В случае, когда узел имеет выводящие характеристики, решение определяется
следующим образом: те компоненты искомого вектора $v^T$, которые не имеют
выводящих характеристик, считаются при помощи сеточно"=характеристического
метода, описанного ранее; остальные уравнения заменяются граничными
соотношениями. После этого, решается полученная СЛАУ, из которой определяются
значения всех компонент вектора $v^T$ в текущем узле.


\subsection{Решение многомерной задачи}

\subsubsection{Расщепление по направлениям}
Идея метода \cite{fedorenko} решения исходной задачи состоит в замене трёхмерной системы
уравнений \ref{matrix_equation} тремя одномерными системами 
\begin{equation}
\frac{\partial}{\partial t}\vec u+\mathbf{A}_x \frac{\partial}{\partial x}\vec u
= 0,
\label{matrix_equation_x}
\end{equation}
\begin{equation}
\frac{\partial}{\partial t}\vec u+\mathbf{A}_y \frac{\partial}{\partial y}\vec u
= 0,
\label{matrix_equation_y}
\end{equation}
\begin{equation}
\frac{\partial}{\partial t}\vec u+\mathbf{A}_z \frac{\partial}{\partial z}\vec u
= 0,
\label{matrix_equation_z}
\end{equation}
\begin{equation}
\frac{\partial}{\partial t}\vec u = \vec f.
\label{matrix_equation_f}
\end{equation}
Эти уравнения решаются последовательно (первым решается уравнение
\ref{matrix_equation_f}, последним -- \ref{matrix_equation_x}) с использованием
на каждом шаге результатов, полученных на предыдущем шаге.


\subsection{Движение сетки}

\todo{Описать движение сетки. Объяснить внятно про конвективные члены.}

\todo{Переписать своими словами выводы Челнокова ниже}

Расчет на лагранжевой сетке из тетраэдров подразумевает, что скорость 
смещения вершин сетки совпадает со скоростью среды. Но в методах 
второго порядка и выше вводятся дополнительные узлы, отличные от
вершин, движение которых задается линейной интерполяцией движений
вершин тетраэдра, в котором они расположены. Даже если скорость
всех вершин является лагранжевой, то узлы, не совпадающие с вершинами, 
смещаются относительно точек среды. Это означает, что при их расчете
необходимо учитывать конвективные члены. Отличие скорости этих узлов
от скорости среды невелико: $O(h^2)$, где $h$ — мелкость сетки. Собственные
значения матриц при поправке на конвекцию также изменятся на $O(h^2)$.

Отличие мест пересечения характеристики до и после поправки со слоем
$t^n$ составит $\tau O(h^h2) = O(\tau^3)$, такого же порядка и отличие в 
восстанавливаемом интерполяцией решении. Поэтому в методе второго порядка можно
не рассматривать отличие движения узлов в центре ребер от лагранжева
— это не понизит степень аппроксимации, а в методах третьего порядков
и выше поправка необходима.

\subsection{Контакты в случае движущихся тел в 3D}

\todo{Написать про выделение контактов}
