\section{Численный метод}

\subsection{Гиперболические свойства определяющей системы уравнений}
Рассмотрим одномерное уравнение вида
\begin{equation}
\frac{\partial}{\partial{t}}\vec{u}^T+\mathbf{A}\frac{\partial}{\partial{x}}\vec{u}^T=\vec{f}^T.
\label{advection_equation}
\end{equation}
Если матрица $\mathbf{A}$ имеет полный набор вещественных собственных значений, 
то такое уравнение называется гиперболическим, и его решения соответствуют 
процессам, которые носят волновой характер. В этом случае справедливо разложение:
$$\mathbf{A}=\mathbf\Omega^{-1}\mathbf\Lambda\mathbf\Omega,$$
где $\mathbf\Omega$ -- матрица, составленная из векторов ${\vec\omega_i}$, а 
$\vec\omega_i$ есть собственные векторы матрицы $\mathbf A$,
удовлетворяющие соотношениям
$$\vec\omega_i\mathbf A=\lambda_i\vec\omega_i,$$
а $\mathbf\Lambda=diag\{\lambda_i\}$ -- диагональная матрица собственных
значений.

Домножив уравнение \ref{advection_equation} слева на $\Omega$, получаем
уравнение
$$\frac{\partial}{\partial t}\Omega{\vec u}^T+
\Lambda\frac{\partial}{\partial x}\Omega{\vec u}^T=\Omega{\vec f}^T,$$
которое после перехода к Римановым инвариантам ${\vec v}^T=\Omega{\vec u}^T$
распадается на $n$ одномерных уравнений вида
\begin{equation}
\frac{\partial}{\partial t}{v_i}+\lambda_i\frac{\partial}{\partial
x}{v_i}={{\tilde f}_i},
\label{advection_equation_splitted}
\end{equation}
где ${{\tilde f}_i}=(\Omega{\vec f}^T)_i$.
Таким образом, решение уравнения \ref{advection_equation} представляется в виде
суммы плоских волн, движущихся со скоростями $\lambda_i$.


\subsection{Расщепление по направлениям}
Идея метода \cite{fedorenko} решения исходной задачи состоит в замене трёхмерной системы
уравнений \ref{matrix_equation} тремя одномерными системами 
\begin{equation}
\frac{\partial}{\partial t}\vec u+\mathbf{A}_x \frac{\partial}{\partial x}\vec u
= 0,
\label{matrix_equation_x}
\end{equation}
\begin{equation}
\frac{\partial}{\partial t}\vec u+\mathbf{A}_y \frac{\partial}{\partial y}\vec u
= 0,
\label{matrix_equation_y}
\end{equation}
\begin{equation}
\frac{\partial}{\partial t}\vec u+\mathbf{A}_z \frac{\partial}{\partial z}\vec u
= 0,
\label{matrix_equation_z}
\end{equation}
\begin{equation}
\frac{\partial}{\partial t}\vec u = \vec f.
\label{matrix_equation_f}
\end{equation}
Эти уравнения решаются последовательно (первым решается уравнение
\ref{matrix_equation_f}, последним -- \ref{matrix_equation_x}) с использованием
на каждом шаге результатов, полученных на предыдущем шаге.


\subsection{Решение одномерной задачи}
Рассмотрим несколько подробнее метод решения одномерной задачи
\begin{equation}
\frac{\partial}{\partial t}v+\lambda \frac{\partial}{\partial x}v = 0.
\label{one_dim_eq}
\end{equation}
Для её решения предлагается использовать сеточно-характеристический метод, суть
которого состоит в следующем. Из того узла $m$ временного слоя $n+1$, в котором
требуется получить решение, опускаются характеристики.
\begin{figure}[h]
\center{\includegraphics[width=0.5\textwidth]{eps/gcm-idea.eps}}
\caption{Принципиальная схема сеточно-характеристического метода.}
\end{figure}
Из точки пересечения характеристики со слоем $n$ значение $v$ переносится в 
точку $\xi^{n+1}_m$:
$$v_i^{n+1}(\xi_m)=v^{n}_i(\xi_m-\lambda_i\tau).$$
Если характеристика не попадает точно в расчётный узел, то применяются различные
методы реконструкции значения в данной точке (в данной работе используется
интерполяция).


\subsection{Расчёт граничных узлов}
Метод, описанный в предыдущем пункте, годится лишь для расчёта внутренних узлов
сетки, т.е. только в том случае, если характеристика, выпущенная из узла, не
выводит за пределы области интегрирования. В случае, когда узел является
внешним, применяется иной подход для решения задачи. Рассматриваемая система
уравнений в граничных узлах области интегрирования имеет не больше трёх
\cite{chelnokov} выводящих характеристик, поэтому для корректной постановки
задачи требуется задание граничных условий для каждого внешнего узла сетки в
количестве, равном числу выводящих характеристик. Граничные условия могут быть
нескольких видов (символы без волны -- для первого тела, с волной -- для второго):
\begin{itemize}
\item{свободная граница
\begin{eqnarray}
\sigma_\tau=\sigma_n=0; \nonumber
\end{eqnarray}}
\item{скольжение тел друг относительно друга 
\begin{eqnarray}
v_n=\tilde{v}_n,\nonumber\\
\sigma_n=\tilde{\sigma}_n,\nonumber\\
\sigma_\tau=\tilde{\sigma}_\tau=0; \nonumber
\end{eqnarray}}
\item{слипание тел
\begin{eqnarray}
v_n=\tilde{v}_n,\nonumber\\
v_\tau=\tilde{v}_\tau.
\end{eqnarray}}
\end{itemize}
В случае, когда узел имеет выводящие характеристики, решение определяется
следующим образом: те компоненты искомого вектора $v^T$, которые не имеют
выводящих характеристик, считаются при помощи сеточно"=характеристического
метода, описанного ранее; остальные уравнения заменяются граничными
соотношениями. После этого, решается полученная СЛАУ, из которой определяются
значения всех компонент вектора $v^T$ в текущем узле.
