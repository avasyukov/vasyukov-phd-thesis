\begin{thebibliography}{99}
\addcontentsline{toc}{section}{СПИСОК ИСПОЛЬЗОВАННЫХ ИСТОЧНИКОВ}
\bibitem{belocerkovsky}Белоцерковский О.М. Численное моделирование в механике
сплошных сред. — М.: Физико-математическая литература. 1994, 442 с.
\bibitem{magomedov}Магомедов К.М., Холодов А.С. Сеточно-характеристические
численные методы. — М.: Наука, 1988, 288 с.

\bibitem{bazhenov}Баженов С.Л., Берлин А.А., Кульков А.А., Ошмян В.Г. Полимерные композиционные материалы. - Долгопрудный: Издательский дом Интеллект, 2010, 352 с.
\bibitem{mills}Миллс Н. Конструкционные пластики - микроструктура, характеристики, применения. - Долгопрудный: Издательский дом Интеллект, 2011, 512 с.
\bibitem{dimitrienko1}Димитриенко Ю.И., Соколов А.П. Современный численный анализ механических свойств композиционных материалов // Известия РАН. Физическая серия – Т. 75, №11. - 2011. – с. 1551-1556. 
\bibitem{dimitrienko2}Димитриенко Ю.И., Соколов А.П. Многомасштабное моделирование упругих композиционных материалов // Математическое моделирование.- Т.24, №5, - 2012.
\bibitem{griffith}Griffith A.A. Phil. Trans. Roy. Soc. London, Ser A, 1920, Vol. 221, P. 163
\bibitem{orowan} Orowan E. Rep. Prog. Phys., 1949, Vol. 12, P. 185-232
\bibitem{regel}Регель В.Г., Слуцкер А.П., Томашевский Э.Е. Кинетическая природа прочности твёрдых тел. М., 1974
\bibitem{deteresa}DeTeresa J., Allen S.R., Farris R.J. and Porter R.S. J. Material Science, 1984. V. 19. P. 57.
\bibitem{rosen}Розен Б.У., Дау Н.Ф. Механика разрушения волокнистых композитов, в кн. Разрушение. Т. 7. Ч. 1. М.: Мир, 1967, С. 300.

\bibitem{landau_lifshits}Ландау Л.Д., Лифшиц Е.М. Теория упругости. М.: Наука, Главная редакция физико-математической литературы, 1965
\bibitem{guz}Гузь А.Н. Бабич И. Устойчивость волокнистых материалов. В кн. Механика материалов. Киев: Наукова Думка, 1982. С. 120.

\bibitem{polilov}Полилов А.Н., Работнов Ю.Н. Механика композит. материалов, 1983. №3. С. 548.
\bibitem{kukudzhanov}Кукуджанов В.Н. Компьютерное моделирование деформирования, повреждаемости и разрушения неупругих материалов и конструкций. М. МФТИ, 2008. - 215 с.

\bibitem{selivanov}Селиванов В.В. Механика разрушения деформируемого тела. М.: Изд-во МГТУ им. Н.Э. Баумана, 1999. - 420 с.

\bibitem{novatsky}Новацкий В. К. Теория упругости. — М. : Мир, 1975, c. 105-107.
\bibitem{sedov}Седов Л. И. Механика сплошной среды. Том 1. — М. : Наука, 1970, с. 143.
\bibitem{rebotnov}Работнов Ю.Н. Механика деформируемого твёрдого тела. — М.: Наука, 1988. — 712 с.
\bibitem{fedorenko}Федоренко Р.П. Введение в вычислительную физику. М.:
Изд-во Моск. физ. -техн. ин-та, 1994, 528 с.
\bibitem{chushkin}Чушкин П.И. Метод характеристик для пространственных сверхзвуковых течений. –  Труды ВЦ АН СССР, 1968, c. 121.
\bibitem{petrov_chelnokov}Петров И.Б., Челноков Ф.Б. Численное исследование волновых процессов и процессов разрушения в многослойных преградах // Журнал вычислительной математики и математической физики – 2003, том 43, N 10, с. 1562-1579.
\bibitem{matyushev_petrov}Matyushev N.G., Petrov I.B. Mathematical Simulation of Deformation and Wave Processes in Multilayered Structures // Computational Mathematics and Mathematical Physics – 2009, Vol. 49, N 9, P. 1615-1621.
\bibitem{petrov_tormasov_holodov}Петров  И.Б., Тормасов А.Г., Холодов А.С. О численном изучении нестационарных процессов в деформируемых средах многослойной структуры // Механика твердого тела – 1989, N 4, с. 89-95.
\bibitem{golubev_kvasov_petrov}Голубев В.И., Квасов И.Е., Петров И.Б. Воздействие природных катастроф на наземные сооружения // Математическое моделирование – 2011, том 23, N 8, с. 46-54.
\bibitem{holodov}Холодов А.С., Холодов Я.А. О критериях монотонности разностных
схем для уравнений гиперболического типа. 
\bibitem{chelnokov}Челноков Ф.Б. Численное моделирование деформационных
процессов в средах со сложной структурой.
\bibitem{agapov_belocerkovsky_petrov}Агапов П.И., Белоцерковский О.М., Петров И.Б. Численное моделирование последствий механического воздействия на мозг человека при черепно-мозговой травме // Журнал вычислительной математики и математической физики – 2006, том 46, N 9, с. 1711-1720.
\bibitem{petrov}Петров И.Б. Волновые и откольные явления в слоистых оболочках конечной толщины // Механика твердого тела – 1986, N 4, с. 118-124.
\bibitem{amenadze}Аменадзе Ю.А. Теория упругости. – М.:Высшая школа, 1976, 272с.
\bibitem{aki_richards}К. Аки, П.Ричардс. Количественная сейсмология : теория и методы. - М. : Мир, 1983.
\bibitem{tischenko}В.И. Тищенко. Характеристики волн Рэлея от глубинных источников. Межведомственный научный сборник "Динамические системы" Выпуск 19.
\bibitem{lamb}H. Lamb, On the propagation of tremors over the surface of an elastic solid, Phil. Trans. Roy. Soc. London A203 (1904), 1–42.
\bibitem{lapwood}Lapwood E. R. The disturbance due to a line source in a semi-infinite elastic medium, Phil. Trans. Roy. Soc. London A242 , 63–100.
\bibitem{viktorov}И.А. Викторов. Физические основы применения ультразвуковых волн Рэлея и Лэмба в технике. -М.: Наука, 1966.
\end{thebibliography}
