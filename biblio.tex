\begin{thebibliography}{99}
\addcontentsline{toc}{section}{Список использованных источников}

% Численные методы -- классика
\bibitem{h}Петров И.Б., Холодов А.С. Численное исследование некоторых динамических задач механики деформируемого твёрдого тела сеточно-характеристическим методом // Журнал вычислительной математики и математической физики, 1984г., Т. 24, № 5, с. 722-739.
\bibitem{petrov_tormasov_holodov}Петров  И.Б., Тормасов А.Г., Холодов А.С. О численном изучении нестационарных процессов в деформируемых средах многослойной структуры // Механика твердого тела – 1989, N 4, с. 89-95.
\bibitem{holodov}Холодов А.С., Холодов Я.А. О критериях монотонности разностных схем для уравнений гиперболического типа. // Журнал вычислительной математики и математической физики, 2006г., Т. 46, № 9, с. 1638-1667.
\bibitem{chelnokov}Челноков Ф.Б. Численное моделирование деформационных процессов в средах со сложной структурой: Дисс. ... канд. физ.-мат. наук – М., 2005
\bibitem{belocerkovsky}Белоцерковский О.М. Численное моделирование в механике сплошных сред. — М.: Физико-математическая литература. 1994, 442 с.
\bibitem{magomedov}Магомедов К.М., Холодов А.С. Сеточно-характеристические численные методы. — М.: Наука, 1988, 288 с.
\bibitem{kukudzhanov}Кукуджанов В.Н. Компьютерное моделирование деформирования, повреждаемости и разрушения неупругих материалов и конструкций. М. МФТИ, 2008. - 215 с.
\bibitem{fedorenko}Федоренко Р.П. Введение в вычислительную физику. М.:Изд-во Моск. физ. -техн. ин-та, 1994, 528 с.
\bibitem{chushkin}Чушкин П.И. Метод характеристик для пространственных сверхзвуковых течений. –  Труды ВЦ АН СССР, 1968, c. 121.
\bibitem{petrov}Петров И.Б. Волновые и откольные явления в слоистых оболочках конечной толщины // Механика твердого тела – 1986, N 4, с. 118-124.
\bibitem{ivanov_kondaurov_petrov_holodov} Иванов В.Д., Кондауров В.И., Петров И.Б., Холодов А.С. Расчет динамического деформирования и разрушения упругопластических тел сеточно-характеристическими методами – Матем. Моделирование № 2:11, 1990, С. 10 – 29
\bibitem{p21}Петров И.Б., Иванов В.Д., Суворова Ю.В. Численное решение двухмерных динамических задач наследственной теории вязкоупругости. // Механика композитных материалов, 1989, №3, с. 419--424.
\bibitem{p27}Петров И.Б., Иванов В.Д., Суворова Ю.В. Расчет волновых процессов в наследственных вязкоупругих средах. // Механика композитных материалов, 1990, №3, с. 447--450.
\bibitem{jp}Сиратори М., Миёси Т., Мацусита Х. Вычислительная механика разрушения. // М.: Мир, 1986. - 334 с.

% Физика -- классика
\bibitem{novatsky}Новацкий В. К. Теория упругости. — М. : Мир, 1975, c. 105-107.
\bibitem{sedov}Седов Л. И. Механика сплошной среды. Том 1. — М. : Наука, 1970, с. 143.
\bibitem{rebotnov}Работнов Ю.Н. Механика деформируемого твёрдого тела. — М.: Наука, 1988. — 712 с.
\bibitem{grigoryan}Григорян С.С. Об основных представлениях динамики грунтов. // Прикладная математика и механика, 1960, Т. XXIV, с. 1057--1072.
\bibitem{vovk}Под.ред. Вовка А.А. Поведение грунтов под действием импульсных нагрузок. // Киев, Наукова думка, 1984. - 279 с.
\bibitem{godunov_phys}Годунов С.К., Роменский Е.И. Элементы механики сплошных сред и законы сохранения. // Новосибирск, Научная книга, 1998. - 280 с.
\bibitem{landau_lifshits}Ландау Л.Д., Лифшиц Е.М. Теория упругости. М.: Наука, Главная редакция физико-математической литературы, 1965.
\bibitem{griffith}Griffith A.A. Phil. Trans. Roy. Soc. London, Ser A, 1920, Vol. 221, P. 163.
\bibitem{orowan} Orowan E. Rep. Prog. Phys., 1949, Vol. 12, P. 185-232.
\bibitem{regel}Регель В.Г., Слуцкер А.П., Томашевский Э.Е. Кинетическая природа прочности твёрдых тел. М., 1974.
\bibitem{deteresa}DeTeresa J., Allen S.R., Farris R.J. and Porter R.S. J. Material Science, 1984. V. 19. P. 57.
\bibitem{selivanov}Селиванов В.В. Механика разрушения деформируемого тела. М.: Изд-во МГТУ им. Н.Э. Баумана, 1999. - 420 с.
\bibitem{parton}Партон В.З. Механика разрушения: от теории к практике. М.: Изд-во ЛКИ, 2007. - 240 с.

% Свои публикации, раскрывающие тему
\bibitem{agapov_vasyukov_petrov} Агапов П.И., Васюков А.В., Петров И.Б. Компьютерное моделирование волновых процессов в покровах мозга при черепно-мозговой травме. // Сборник научных трудов <<Процессы и методы обработки информации>>. М.: МФТИ, 2006. С. 154--163.
\bibitem{a2} Агапов П.И., Васюков А.В. Компьютерное моделирование биомеханических процессов в покровах мозга. // Труды 49-й научной конференции МФТИ <<Современные проблемы фундаментальных и прикладных наук>>: Часть VII. Управление и прикладная математика. М.: <<Солар>>, 2007. С. 41--42.
\bibitem{a3} Васюков А.В., Петров И.Б. Компьютерное моделирование биомеханических процессов в покровах мозга при динамическом нагружении. Сравнение механических моделей. // Сборник научных трудов <<Моделирование процессов обработки информации>>. М.: МФТИ, 2007. С. 67--76.
\bibitem{a4} Васюков А.В., Петров И.Б. О разработке параллельной версии сеточно-характеристического метода для трехмерных уравнений механики деформируемого твердого тела. // Сборник научных трудов <<Модели и методы обработки информации>>. М.: МФТИ, 2009. С. 13--17.
\bibitem{a5} Васюков А.В., Петров И.Б., Стрижевская А.Д. Компьютерное моделирование волновых процессов в гидроупругих средах сеточно"=характеристическим методом. // Сборник научных трудов <<Модели и методы обработки информации>>. М.: МФТИ, 2009. С. 18--22.
\bibitem{a6} Васюков А.В., Петров И.Б., Черников Д.В. О сеточно"=характеристическом численном методе на неструктурированных сетках для задач механики деформируемого твердого тела в случае трех пространственных переменных. // Сборник научных трудов <<Информационные технологии: модели и методы>>. М.: МФТИ, 2010. С. 52--57.
\bibitem{a7} Болоцких Ю.В., Васюков А.В., Петров И.Б. О численном решении некоторых задач биомеханики. // Сборник научных трудов <<Информационные технологии: модели и методы>>. М.: МФТИ, 2010. С. 58--64.
\bibitem{a8} Васюков А.В., Петров И.Б. Моделирование механических факторов черепно-мозговых травм сеточно-характеристическим численным методом. // Вестник Российского государственного университета им. И.Канта. Калининград: БФУ им. И.Канта, 2010, вып.10. С. 42--51.
\bibitem{a9} Васюков А.В., Петров И.Б. Компьютерное моделирование последствий механических черепно-мозговых травм. // Информационные технологии, 2011, №5. С. 58--62.
\bibitem{a10} I. Petrov, Y. Bolotskikh and A. Vasyukov. Modeling of Dynamic Problems in Biomechanics. // Math. Model. Nat. Phenom. 2011, Vol. 6, No. 7, pp. 70--81.
\bibitem{a11} Igor Petrov, Alexey Vasyukov, Dmitry Chernikov, Yulia Bolotskikh. Modeling of dynamic problems in biomechanics using HPC clusters. // The 8th Congress of the International Society for Analysis, its Applications, and Computation. Book of Abstracts. -- M.: PFUR, 2011 -- P. 342.
\bibitem{a12} Васюков А.В. О решении задач динамической прочности трубопроводов под давлением с использованием параллельной версии сеточно-характеристического численного метода на неструктурированных сетках. // Труды 54-й научной конференции МФТИ <<Проблемы фундаментальных и прикладных естественных и технических наук в современном информационном обществе>>. Том 2. Управление и прикладная математика. М.: МФТИ, 2011. С. 60--61.

% Разное по композитам
\bibitem{bazhenov}Баженов С.Л., Берлин А.А., Кульков А.А., Ошмян В.Г. Полимерные композиционные материалы. - Долгопрудный: Издательский дом Интеллект, 2010, 352 с.
\bibitem{mills}Миллс Н. Конструкционные пластики - микроструктура, характеристики, применения. - Долгопрудный: Издательский дом Интеллект, 2011, 512 с.
\bibitem{bahvalov} Бахвалов Н. С., Панасенко Г. П. Осреднение процессов в периодических средах — математические задачи механики композиционных материалов. 1984
\bibitem{dimitrienko1}Димитриенко Ю.И., Соколов А.П. Современный численный анализ механических свойств композиционных материалов // Известия РАН. Физическая серия – Т. 75, №11. - 2011. – с. 1551-1556. 
\bibitem{dimitrienko2}Димитриенко Ю.И., Соколов А.П. Многомасштабное моделирование упругих композиционных материалов // Математическое моделирование.- Т.24, №5, - 2012.
\bibitem{rosen}Розен Б.У., Дау Н.Ф. Механика разрушения волокнистых композитов, в кн. Разрушение. Т. 7. Ч. 1. М.: Мир, 1967, С. 300.
\bibitem{guz}Гузь А.Н. Бабич И. Устойчивость волокнистых материалов. В кн. Механика материалов. Киев: Наукова Думка, 1982. С. 120.
\bibitem{polilov}Полилов А.Н., Работнов Ю.Н. Механика композит. материалов, 1983. №3. С. 548.

% Волны в среде
\bibitem{amenadze}Аменадзе Ю.А. Теория упругости. – М.:Высшая школа, 1976, 272с.
\bibitem{aki_richards}К. Аки, П.Ричардс. Количественная сейсмология : теория и методы. - М. : Мир, 1983.
\bibitem{tischenko}В.И. Тищенко. Характеристики волн Рэлея от глубинных источников. Межведомственный научный сборник "Динамические системы" Выпуск 19.
\bibitem{lamb}H. Lamb, On the propagation of tremors over the surface of an elastic solid, Phil. Trans. Roy. Soc. London A203 (1904), 1–42.
\bibitem{lapwood}Lapwood E. R. The disturbance due to a line source in a semi-infinite elastic medium, Phil. Trans. Roy. Soc. London A242 , 63–100.
\bibitem{viktorov}И.А. Викторов. Физические основы применения ультразвуковых волн Рэлея и Лэмба в технике. -М.: Наука, 1966.

% Смежные публикации примерно о том же
\bibitem{biomech_first} Агапов П.И., Обухов А.С., Петров И.Б., Челноков Ф.Б. Компьютерное моделирование биомеханических процессов сеточно-характеристическим методом // Управление и обработка информации: модели процессов. Сб. ст. – М.: МФТИ, 2001 и сайт http://www.cs.mipt.ru/index.php?id=27
\bibitem{agapov_petrov_komp_i_mozg} Агапов П. И., Петров И. Б. Расчет областей повреждения мозга при черепно-мозговой травме // Сборник “Компьютер и мозг. Новые технологии” – М.: Наука, 2005, С. 28 – 38.
\bibitem{agapov_2005} Агапов П. И. Анализ результатов численного моделирования черепно-мозговой травмы // Процессы и методы обработки информации: Сб. ст. – М.: МФТИ, 2005. – С. 186 – 193.
\bibitem{agapov_diser} Агапов П.И. Численное моделирование механических факторов черепно-мозговой травмы: Дисс. ... канд. физ.-мат. наук – М., 2005
\bibitem{agapov_belocerkovsky_petrov}Агапов П.И., Белоцерковский О.М., Петров И.Б. Численное моделирование последствий механического воздействия на мозг человека при черепно-мозговой травме // Журнал вычислительной математики и математической физики – 2006, том 46, N 9, с. 1711-1720.
\bibitem{belocerkovsky_agapov_petrov} Белоцерковский О.М., Агапов П.И., Петров И.Б. Моделирование последствий черепно-мозговой травмы // Медицина в зеркале информатики – М.: Наука, 2008, С. 113 – 124.
\bibitem{b}Квасов И.Е., Петров И.Б., Челноков Ф.Б. Расчет волновых процессов в неоднородных пространственных конструкциях // Матем. Моделирование, 21:5, 2009г., с. 3-9.
\bibitem{g}Иванов В.Д., Кондауров В.Н., Холодов А.С. Расчет динамического деформирования и разрушения упругопластических тел сеточно-характеристическими методами. // Математическое моделирование, 2003, Т. 15, № 10.
\bibitem{petrov_chelnokov}Петров И.Б., Челноков Ф.Б. Численное исследование волновых процессов и процессов разрушения в многослойных преградах // Журнал вычислительной математики и математической физики – 2003, том 43, N 10, с. 1562-1579.
\bibitem{matyushev_petrov}Matyushev N.G., Petrov I.B. Mathematical Simulation of Deformation and Wave Processes in Multilayered Structures // Computational Mathematics and Mathematical Physics – 2009, Vol. 49, N 9, P. 1615-1621.
\bibitem{golubev_kvasov_petrov}Голубев В.И., Квасов И.Е., Петров И.Б. Воздействие природных катастроф на наземные сооружения // Математическое моделирование – 2011, том 23, N 8, с. 46-54.
\bibitem{agapov_petrov_chelnokov_2002} Агапов П.И., Петров И.Б., Челноков Ф.Б. Численное исследование задач механики деформируемого твёрдого тела в неоднородных областях интегрирования // Обработка информации и моделирование: Сб. ст. – М.: МФТИ, 2002 и сайт http://www.cs.mipt.ru/index.php?id=27.

% ЧМТ и другая биомеханика
\bibitem{zhou} Zhou C., Khalil T. B., King A. I. A new model comparing impact responses of the homogeneous and inhomogeneous human brain // 39th Stapp Car Crash Conf. / Society of Automotive Engineers. – 1995. – Pp. 121 – 137.
\bibitem{kuijpers} Kuijpers A. H., Claessens M. H., Sauren A. A. The influence of different boundary conditions on the response of the head to impact: a two-dimensional finite element study // J. Neurotrauma. – Vol. 12, no. 4. – Pp. 715 – 724.
\bibitem{chu} Chu С., Lin M., Huang H.M., Lee M.C. Finite element analysis of cerebral contusion // J. Biomechanics. – 1994. – Vol. 27. – Pp. 187–194.
\bibitem{claessens} Claessens M. H. A. Finite Element Modeling of the Human Head under Impact Conditions: Ph.D. thesis / Eindhoven University of Technology. – 1997.
\bibitem{nahum} Nahum A. M., Smith R. W., Ward C. C. Intracranial pressure dynamics during head impact // 21th Stapp Car Crash Conf. / Society of Automotive Engineers. – 1977. 
\bibitem{ueno} Ueno K., Melvin J.W., Lundquist E., Lee M.C. Two dimensional finite element analysis of human brain impact responses: Application of scaling law. // BED vol. 13 ASME. – 1989. – Pp. 123 – 124. 
\bibitem{ruan} Ruan J. S., Khalil T., King A. I. Human head dynamic response to side impact by finite element modeling // J. Biomechanical Engineering. – Vol. 113, no. 3. – Pp. 276 – 283. 
\bibitem{willinger} Willinger R. Modal analysis of a finite element model of the head // IRCOBI Conf. – Verona: 1992. – Pp. 283 – 297.
\bibitem{petrov_biomech_2003} Петров И.Б. О численном моделировании биомеханических процессов в медицинской практике. // Информационные технологии и вычислительные системы – 2003, No 1-2, С. 102 – 111.
\bibitem{dlima} D’Lima D. Realistic Simulation Probes Biomechanics of Knees // сайт http://www.designworldonline.com/articles/4317/314/Realistic-Simulation-Probes-Biomechanics-of-Knees.aspx
\bibitem{begun} Бегун П.И., Афонин П.Н. Моделирование в биомеханике // М.: Высшая школа -- 2004
\bibitem{anatomy_atlas} Рохен Й., Йокочи Ч., Лютьен-Дреколль Э. Большой атлас по анатомии – Сайт http://www.medbook.net.ru

% Параллельность и разные тулзы
\bibitem{gergel} Гергель В.П. Теория и практика параллельных вычислений: учебное пособие – М.: Интернет-Университет Информационных Технологий; БИНОМ. Лаборатория знаний, 2007
\bibitem{gmsh} Geuzaine C. and Remacle J.-F. Gmsh: a three-dimensional finite element mesh generator with built-in pre- and post-processing facilities. International Journal for Numerical Methods in Engineering, Volume 79, Issue 11, pages 1309-1331, 2009

\end{thebibliography}
