\begin{center}\section*{Введение}\end{center}
\addcontentsline{toc}{section}{Введение}

Основанием для проведения данной НИР являются ФЦП «Развитие гражданской авиационной техники России на 2002-2010 годы и на период до 2015 года», а также дополнительное соглашение №2 к Государственному контракту от 03.05.2011 г. № 11411.1003800.18.035, заключенному между Министерством промышленности и торговли РФ с ФГУП «ЦАГИ» на выполнение НИР «Комплексные экспериментальные и расчетные исследования прочности, ресурса и аэроупругости силовых и несиловых авиационных конструкций из композиционных материалов» (шифр «КМ-планер 2011»).

В данной работе рассматривается метод численного моделирования низкоскоростного
удара по инженерной конструкции, выполненной из композиционных материалов.

Использование композиционных материалов открывает новые перспективы в авиастроении 
благодаря сочетанию лёгкости и высокой прочности. На сегодняшний день рассматривается 
возможность применения композиционных материалов в ответственных силовых конструкциях 
оперения, крыла и фюзеляжа, что позволит значительно снизить массу самолёта. Благодаря этому 
станет возможной реализация новых конструктивно-силовых схем и компоновок летательных 
аппаратов и улучшения их характеристик.

В связи с этим важными задачами являются как разработка новых усовершенствованных 
композиционных материалов, так и создание методик и норм проверки их прочностных характеристик 
и надёжности в эксплуатации. На основании фундаментальных исследований свойств композиционных 
материалов необходимо выработать подходы к проектированию новых силовых схем.

Данная работа посвящена решению одной из актуальных прикладных задач, связанных с 
прочностными испытаниями композиционных материалов, -- изучению поведения материала при 
динамической нагрузке. В силу анизотропности свойств композиционные материалы после
действия нагрузки могут заметно терять прочность даже при отсутствии видимых поврежедний.
Это обусловлено появлением микротрещин, которые впоследствии, объединяясь,
превращаются в макротрещины. Так, возникающее после нагрузки расслоение
материала может быть визуально не заметно, хотя делает образец непригодным к
дальнейшему использованию.

Возникновение данных повреждений носит выраженный волновой характер. 
Динамическое воздействие вызывает распространение упругих волн в образце. В случае 
композиционного материала множественные переотражения волн от внутренних контактных 
границ между слоями создают сложную волновую картину. Интерференция прямых, отражённых 
и преломлённых волн формирует итоговые области максимальных нагрузок в конструкции.

В связи с этим для моделирования необходимо использовать численный метод решения системы 
уравнений механики деформируемого твёрдого тела, позволяющий получить полную волновую 
картину с высоким временным и пространственным разрешением с учётом влияния контактных 
границ. Указанными свойствами обладает сеточно-характеристический численный метод, 
применяемый в данной работе.
