\section*{Введение}
\addcontentsline{toc}{section}{Введение}

\subsection*{Актуальность темы}

Задачи деформаций и разрушения сложных конструкций представляют особый интерес для многих областей техники. Механика разрушения ставит множество как академических проблем, касающихся механизмов разрушения материалов различных типов, так и инженерных задач, связанных с требованиями обеспечить необходимые уровни надежности различных изделий.

Решение задач прочности конструкций сложной формы и реологии при непростых условиях нагружения трудно представить без применения компьютерного моделирования и эффективных численных методов. На сегодняшний день наиболее широкое распространение для данного класса задач получил метод конечных элементов (МКЭ). Основные параметры, используемые для описания условий разрушения в расчётах прочности методом конечных элементов, -- коэффициент интенсивности напряжений, J-интеграл, раскрытие в вершине трещины. Применение МКЭ и данных критериев позволяет эффективно решать статические задачи прочности.

Однако, для определения деформаций и повреждений в сложных конструкциях при динамической нагрузке требуется разработка методов, учитывающих волновые процессы при соударении. Особенно актуальна эта задача для многослойных и неоднородных материалов, в которых итоговая сложная картина повреждений формируется в результате множественных взаимодействий упругих и пластических волн как с внешними, так и с внутренними контактными границами. Ярким примером таких материалов являются современные композиты.

На сегодняшний день композиционные материалы активно внедряются во многих областях техники. Их использование открывает новые перспективы в авиастроении, космической отрасли, машиностроении и других отраслях благодаря сочетанию лёгкости и высокой прочности. В том числе активно рассматривается возможность применения композиционных материалов в ответственных силовых конструкциях оперения, крыла и фюзеляжа самолёта, что позволит значительно снизить массу конструкции. Благодаря этому станет возможной реализация новых конструктивно-силовых схем и компоновок летательных аппаратов и улучшения их характеристик.

В связи с этим важными задачами являются как разработка новых усовершенствованных 
композиционных материалов, так и создание методик и норм проверки их прочностных характеристик 
и надёжности в эксплуатации. Существующие методы проверки монолитных изделий из металлов и сплавов оказываются неэффективны для композитов в силу их сложной внутренней структуры.

Данная работа непосредственно связана с одной из актуальных прикладных задач 
прочностных испытаний композиционных материалов -- изучение поведения материала при 
динамической нагрузке. В силу анизотропности свойств композиционные материалы после
действия нагрузки могут заметно терять прочность даже при отсутствии видимых поврежедний.
Это обусловлено появлением микротрещин, которые впоследствии, объединяясь,
превращаются в макротрещины. Так, возникающее после нагрузки расслоение
материала может быть визуально не заметно, хотя делает образец непригодным к
дальнейшему использованию.

Разрушение композиционных материалов может происходить как в объеме (при сжатии, растяжении, 
сдвиге), так и на контактных границах между матрицей и наполнителем. В зависимости от типа 
нагрузки разрушение может носить деформационный или волновой характер. 
Динамическое воздействие вызывает распространение упругих волн в образце. В случае 
композиционного материала множественные переотражения волн от внутренних контактных 
границ между слоями создают сложную волновую картину. Интерференция прямых, отражённых 
и преломлённых волн формирует итоговые области максимальных нагрузок в конструкции.

В связи с этим для моделирования необходимо использовать численный метод решения системы 
уравнений механики деформируемого твёрдого тела, позволяющий получить полную волновую 
картину с высоким временным и пространственным разрешением с учётом влияния контактных 
границ. Указанными свойствами обладает сеточно-характеристический численный метод, 
применяемый в данной работе.

Для моделирования реальных инженерных конструкций необходимо разработать и реализовать численные методы, позволяющие выполнять расчёты в областях сложной геометрии. Для решения задач большой размерности требуется параллельная реализация используемых численных методов, обладающая высокой эффективностью при использовании на современных высокопроизводительных вычислительных комплексах.

\subsection*{Цели работы}

\begin{enumerate}
\item Разработка математических моделей для задачи низкоскоростного удара по инженерной конструкции, выполненной из композиционных материалов.
\item Разработка сеточно-характеристического метода, позволяющего выполнять расчёты на сетке из тетраэдров с шагом $\tau > h / \lambda$ (здесь $\tau$ -- шаг по времени, $h$ -- минимальное расстояние от узла сетки до соседних узлов, $\lambda$ -- максимальное по модулю собственное число упределяющей системы уравнений).
\item Разработка параллельной версии сеточно-характеристического метода с явным выделением контактных границ, обеспечивающей высокую эффективность при использовании на современных высокопроизводительных вычислительных комплексах.
\item Создание комплекса программ для решения задач предметной области. Интеграция комплекса с существующими сторонними программами задания геометрии объектов и визуализации результатов расчётов, являющимися стандартом де-факто среди инженеров-практиков.
\item Исследование волновых процессов в средах сложной структуры, численное решение задач об объёмных волнах, поверхностных волнах, волнах на контактной границе.
\item Исследование волновых процессов в элементе композитной обшивки и силового кессона крыла самолёта, приводящих к повреждениям конструкции при низкоскоростном ударе.
\end{enumerate}

\subsection*{Научная новизна}

\begin{enumerate}

\item Разработан метод численного моделирования на неструктурированной сетке действия низкоскоростного удара на конструкцию сложной формы в трёхмерной постановке. Разработанный метод позволяет проводить моделирование волновых процессов в конструкции при динамическом внешнем воздействии с учетом взаимодействия волновых фронтов, влияния внешних и внутренних границ, различия реологических свойств слоёв. Особенностью метода является возможность выполнять расчёты с шагом $\tau > h / \lambda$ в трёхмерной постановке. Разработанный метод исследован на аппроксимацию и устойчивость. Проведено тестирование реализации метода.

\item Разработанный сеточно-характеристический метод реализован в виде параллельного вычислительного комплекса, позволяющего выполнять моделирование как на стандартном оборудовании, так и на современных высокопроизводительных вычислительных комплексах.

\item Выполнено исследование волновых процессов в многослойных средах различной структуры, моделирующих панель из полимерного композиционного материала. Исследование включает в себя как аналитическое, так и численное изучение процессов, протекающих в многослойной среде при динамическом нагружении. Получены поля скоростей и напряжений, области потенциальных разрушений различных типов, обусловленные распространением и взаимодействием волновых фронтов в материале.

\item Выполнено численное моделирование натурного эксперимента по динамическому нагружению элемента композитной обшивки и силового кессона крыла самолёта. Проведены расчеты для двух постановок эксперимента -- удар по отдельному элементу обшивки и удар по элементу обшивки со стрингером. Для задачи со стрингером рассмотрены постановки с центральным и нецентральным ударом. Проведен анализ причин разрушения композиционных авиационных материалов. Для всех постановок получены области концентрации напряжений, вызванные волновыми процессами в ходе соударения. Определены зоны потенциальных повреждений конструкции, обусловленные разными механизмами разрушения материала. Для элемента обшивки без стрингера размер разрушенной области составляет 50-60 мм, для элемента обшивки со стрингером 25-30 мм при центральном ударе и 20-25 мм при нецентральном ударе.

\item Получено, что наличие стрингера существенно разгружает элемент обшивки при динамическом воздействии и уменьшает размер потенциально повреждённых областей. Данный результат важен, так как при действии статической нагрузки наличие стрингера напротив вызывает концентрацию напряжений и приводит к разрушению при меньшей силе воздействия.

\item Разработанный численный метод применен для решения ряда задач биомеханики. Получены области потенциальных повреждений тканей организма человека в задачах о черепно-мозговой травме, о динамическом нагружении коленного сустава и об ударе по торсу в защитной конструкции.

\end{enumerate}

\subsection*{Практическая ценность}

Результаты численного моделирования действия низкоскоростного удара на конструкцию из полимерного композиционного материала могут быть использованы для экспериментальной проверки предложенных математических моделей и численного метода. В работе сформулированы критерии для сравнения численного и натурного эксперимента, учитывающие механические свойства распространённых полимерных матриц.

После экспериментальной верификации разработанные модели и методы могут быть использованы при создании методик и норм проверки прочностных характеристик композиционных материалов.

Полученные результаты по взаимодействию упругой волны с разрушенной областью конструкции могут быть использованы при разработке методов неразрушающего контроля состояния изделий из композиционных материалов.

Кроме того, разработанный параллельный программный комплекс может быть использован для моделирования динамического воздействия на комплексные силовые конструкции из композиционных материалов в тех случаях, когда проведение натурных испытаний затруднительно.

Полученные результаты в части задач биомеханики могут быть использованы при разработке защитного снаряжения различных видов.

Работа поддержана рядом государственных и коммерческих грантов.

\begin{enumerate}

\item Федеральное государственное унитарное предприятие <<Российский Федеральный Ядерный Центр -- Всероссийский научно"=исследовательский институт экспериментальной физики (ФГУП <<РФЯЦ-ВНИИЭФ>>)>>. НИР5. <<Разработка физико-математических моделей, алгоритмов и эффективных методов решения задач механики сплошных сред на супер-ЭВМ>>;

\item Грант РФФИ 10-01-92654-ИНД\_а <<Математическое моделирование сложных задач на высокопроизволительных вычислительных системах>>, 2010--2011гг.

\item Грант РФФИ 11-01-00723-а <<Разработка численных методов моделирования динамических задач биомеханики на современных высокопроизводительных вычислительных системах>>, 2011--2013гг.

\item Грант РФФИ 10-01-00572-а <<Разработка алгоритмического обеспечения и вычислительных методов для численного решения задач динамики деформируемых сред на многопроцессорных ЭВМ нового поколения>>, 2010--2012гг.

\end{enumerate}

\subsection*{Публикации}

Научные результаты диссертации опубликованы в 12 работах (\cite{agapov_vasyukov_petrov} - \cite{a12}), из которых две (\cite{a8} и \cite{a9}) -- в изданиях, рекомендованных ВАК для публикации основных результатов диссертации.

\subsection*{Апробация}

Результаты работы были доложены, обсуждены и получили одобрение специалистов на следующих научных конференциях:

\begin{enumerate}
\item Научные конференции Московского физико-технического института <<Проблемы фундаментальных и прикладных, естественных и технических наук в современном информационном обществе>> (МФТИ, Долгопрудный, 2006--2011);
\item I международная конференция <<Математические модели и численные методы в биоматематике>> (Институт вычислительной математики РАН, Москва, 2010);
\item II международная конференция <<Математические модели и численные методы в биоматематике>> (Институт вычислительной математики РАН, Москва, 2011);
\item Расширенный семинар <<Вычислительная физика: алгоритмы, методы и результаты>> (представительство Института космических исследований РАН, Таруса, 2011);
\item The 8th Congress of the International Society for Analysis, its Applications, and Computation (ISAAC 2011) (Российский университет дружбы народов, Москва, 2011);
\item Российско-индийский семинар <<Новые достижения математического моделирования>> (Институт автоматизации проектирования РАН, Москва, 2011);
\item Международный авиационно-космический семинар им. С.М. Белоцерковского (Центральный аэрогидродинамический институт имени профессора Н.Е. Жуковского, Москва, 2012).
\end{enumerate}

Результаты работы были доложены, обсуждены и получили одобрение специалистов на научных семинарах в следующих организациях:
\begin{enumerate}
\item Центральный аэрогидродинамический институт имени профессора Н.Е. Жуковского (Москва--Жуковский, 2011, 2012);
\item Научно-исследовательский институт природных газов и газовых технологий – Газпром ВНИИГАЗ (Москва, 2011);
\item Институт вычислительной математики РАН (Москва, 2010, 2011);
\item Институт автоматизации проектирования РАН (Москва, 2011).
\end{enumerate}

\subsection*{Структура и объем диссертации}

Диссертация состоит из введения, пяти глав, заключения и списка использованных источников. Общий объем диссертации составляет 200 страниц. Список использованных источников содержит ссылки на 79 публикаций.

\subsection*{Содержание работы}

В диссертационной работе рассматриваются задачи механики деформируемого твердого тела, связанные с деформациями и повреждениями сложных конструкций при действии динамической внешней нагрузки. Для решения указанных задач конструируется сеточно-характеристический метод на неструктурированной сетке для случая трех пространственных переменных и низкого качества расчетной сетки. Метод реализуется в виде параллельного программного комплекса. С применением разработанного метода исследуются волновые процессы в многослойных средах, рассчитывается задача о низкоскоростном ударе по элементу композитной обшивки и силового кессона крыла самолёта, выполняются тестовые расчета ряда задач биомеханики.

Во введении обсуждается актуальность темы диссертации, описываются основные цели работы, формулируется новизна работы и ее практическая ценность, указываются положения, выносимые на защиту.

В первой главе рассматриваются математические модели деформируемого твердого тела. Дается обзор различных моделей линейно"=упругого тела, упруго"=пластического тела, вязко"=упругого тела, вязко"=упруго"=пластического тела. Обсуждаются подходы к моделированию композиционных материалов, обладающих несколькими уровнями структуры.

Для различных математических моделей приводится матричная форма записи системы определяющих уравнений в частных производных. Исследуются свойства матрицы общего вида $A_q$, возникающей при преобразовании координат. Получено выражение для изменения собственных чисел матрицы при преобразовании координат, задающее ограничение на допустимый шаг по времени.

Также в первой главе рассматриваются различные модели разрушения, реализованные в дальнейшем в вычислительном комплексе.

Вторая глава посвящена сеточно-характеристическому численному методу. Рассматриваются гиперболические свойства определяющей системы уравнений и конструирование метода на их основе. Обсуждаются численные методы разных порядков точности на структурированных и неструктурированных сетках, применение гибридизации для повышения качества численного решения, расчет внутренних, граничных и контактных узлов сетки. Приводится описание алгоритма перехода от одномерной задачи к многомерной с использованием расщепления разных порядков точности. Отдельно рассматривается алгоритм выделения контактных границ и зон контакта тел в трехмерном случае при взаимном движении тел и деформациях сетки.

Также во второй главе описывается конструирование метода для работы на неструктурированной сетке низкого качества, позволяющего выполнять расчет с шагом по времени $\tau > h / \lambda$. Приводится исследование метода на аппроксимацию и устойчивость, результаты тестирования метода на модельных задачах. Также обсуждается обобщение метода на трехмерный случай.

В заключительной части второй главы рассматривается алгоритм параллельной версии разработанного сеточно-характеристического численного метода и архитектура параллельного программного комплекса. Приводятся результаты тестирования производительности и масштабирования параллельной версии.

В третьей главе приведено исследование волновых процессов в средах сложной структуры. Получены как аналитические, так и численные решения. Получены решения для объемных волн (S-волна, P-волна), поверхностных волн (волны Рэлея и Лэмба, отражение сферической волны от свободной границы), волн на контактной границе (преломление на границе, волны Стоунли и Лява). Выполняется расчет многослойной конструкции, моделирующей композиционный материал, обсуждается связь волновых процессов с различными критериями разрушения. Также в данной главе рассматривается задача о вторичном ударе и взаимодействии упругой волны с зоной материала, разрушенного после первого импульса нагрузки.

В четвертой главе решается задача о низкоскоростном динамическом нагружении элемента композитной обшивки и силового кессона крыла самолёта. Проведены расчеты для двух постановок эксперимента -- удар по отдельному элементу обшивки и удар по элементу обшивки со стрингером. Для задачи со стрингером рассмотрены постановки с центральным и нецентральным ударом. Для всех постановок получены области концентрации напряжений, вызванные волновыми процессами в ходе соударения. Определены зоны потенциальных повреждений конструкции, обусловленные разными механизмами разрушения материала. Для элемента обшивки без стрингера размер разрушенной области составляет 50-60 мм, для элемента обшивки со стрингером 25-30 мм при центральном ударе и 20-25 мм при нецентральном ударе. Получено, что наличие стрингера существенно разгружает элемент обшивки при динамическом воздействии и уменьшает размер потенциально повреждённых областей. Данный результат важен, так как при действии статической нагрузки наличие стрингера напротив вызывает концентрацию напряжений и приводит к разрушению при меньшей силе воздействия.

В пятой главе разработанные методы применяются для решения ряда задач биомеханики. Выполняется численное моделирование задач о черепно-мозговой травме, о динамическом нагружении согнутого коленного сустава, об ударе по торсу в защитной конструкции. Для всех задач получены картины распространения возмущения от удара и зоны концентрации напряжений, в которых возможны повреждения тканей и внутренних органов.

В заключении приводятся основные результаты и выводы, полученные в ходе выполнения работы.


\subsection*{Основные результаты и выводы диссертации}

\begin{enumerate}

\item Разработана математическая модель панели из полимерного композиционного материала для задачи о низкоскоростном ударе по элементу композитной обшивки и силового кессона крыла самолёта. Модель может быть использована в том числе для более сложных инженерных конструкций, выполненных из композиционных материалов.

\item Разработан сеточно-характеристический метод на сетке из тетраэдров для моделирования удара по конструкции сложной формы в трёхмерной постановке. Разработанный метод позволяет проводить моделирование волновых процессов в конструкции при динамическом внешнем воздействии с учетом взаимодействия волновых фронтов, влияния внешних и внутренних границ, различия реологических свойств слоёв. Особенностью метода является возможность выполнять расчёты с шагом $\tau > h / \lambda$ (здесь $\tau$ -- шаг по времени, $h$ -- минимальное расстояние от узла сетки до соседних узлов, $\lambda$ -- максимальное по модулю собственное число упределяющей системы уравнений). Разработанный метод исследован на аппроксимацию и устойчивость. Проведено тестирование реализации метода.

\item Разработанный сеточно-характеристический метод реализован в виде параллельного вычислительного комплекса, позволяющего выполнять моделирование как на стандартном оборудовании, так и на современных высокопроизводительных вычислительных комплексах. Для явного выделения контактных границ при параллельном расчете разработан параллельный алгоритм детектора столкновений.

\item Реализованный параллельный вычислительный комплекс интегрирован с существующими программами задания геометрии объектов (Gmsh, Tetgen, Ani3D) и визуализации результатов расчётов (Paraview, Mayavi), являющимися стандартом де-факто среди инженеров-практиков.

\item Выполнено исследование волновых процессов в многослойных средах сложной структуры, моделирующих панель из полимерного композиционного материала. Исследование включает в себя как аналитическое, так и численное изучение процессов, протекающих в многослойной среде при динамическом нагружении (задачи об объёмных волнах, поверхностных волнах, волнах на контактной границе). Получены поля скоростей и напряжений, области потенциальных разрушений различных типов, обусловленные распространением и взаимодействием волновых фронтов в материале.

\item Выполнено численное моделирование натурного эксперимента по динамическому нагружению элемента композитной обшивки и силового кессона крыла самолёта. Проведены расчеты для двух постановок эксперимента -- удар по отдельному элементу обшивки и удар по элементу обшивки со стрингером. Для задачи со стрингером рассмотрены постановки с центральным и нецентральным ударом. Проведен анализ причин разрушения композиционных авиационных материалов. Для всех постановок получены области концентрации напряжений, вызванные волновыми процессами в ходе соударения. Определены зоны потенциальных повреждений конструкции, обусловленные разными механизмами разрушения материала. Для элемента обшивки без стрингера размер разрушенной области составляет 50-60 мм, для элемента обшивки со стрингером 25-30 мм при центральном ударе и 20-25 мм при нецентральном ударе. Получено, что наличие стрингера существенно разгружает элемент обшивки при динамическом воздействии и уменьшает размер потенциально повреждённых областей. Данный результат важен, так как при действии статической нагрузки наличие стрингера напротив вызывает концентрацию напряжений и приводит к разрушению при меньшей силе воздействия.

\item Дополнительно разработанные математические модели и численный метод были применены для решения ряда задач биомеханики. Получены области потенциальных повреждений тканей организма человека в задачах о черепно-мозговой травме, о динамическом нагружении коленного сустава и об ударе по торсу в защитной конструкции.

\end{enumerate}


\subsection*{Положения, выносимые на защиту}

На защиту выносятся:

\begin{enumerate}

\item Разработанный сеточно-характеристический метод численного моделирования задач механики деформируемого твердого тела на неструктурированной сетке в случае трех пространственных переменных. Основной особенностью метода является возможность выполнять расчёты с шагом $\tau > h / \lambda$.

\item Разработанный алгоритм параллельной версии сеточно"=характеристического метода на неструктурированных сетках с явным выделением контактных границ для задачи многих тел. Алгоритм реализован в виде параллельного вычислительного комплекса. Комплекс позволяет выполнять расчеты задач механики деформируемого твердого тела в трехмерной постановке на неструктурированных сетках в областях интегрирования сложной формы при наличии динамических контактных взаимодействий и конечных деформаций.

\item Математическая модель панели из полимерного композиционного материала для задачи о низкоскоростном ударе по элементу композитной обшивки и силового кессона крыла самолёта.

\item Анализ причин внутреннего повреждения элемента композитной обшивки и силового кессона крыла самолета при низкоскоростном ударе.

\item Математические модели покровов головного мозга человека, коленного сустава, торса человека в защитной конструкции.

\end{enumerate}


\subsection*{Личный вклад соискателя в работах с соавторами}

В части моделей соискателем разработана математическая модель панели из полимерного композиционного материала для задачи о низкоскоростном ударе по элементу композитной обшивки и силового кессона крыла самолёта. Также выполнено исследование свойств матрицы общего вида $A_q$, возникающей при программной реализации модели.

В части численных методов соискателем предложен и реализован сеточно-характеристический метод, позволяющий выполнять расчёты с шагом $\tau > h / \lambda$ в трёхмерной постановке. Выполнено исследование разработанного метода на аппроксимацию и устойчивость. Проведено тестирование реализации метода.

В части программной реализации метода и разработки параллельного вычислительного комплекса соискателем разработан и реализован алгоритм параллельной версии численного метода, предложен алгоритм параллельного детектора столкновений, выполнена интеграция программного комплекса с программами задания геометрии объектов (Gmsh, Tetgen, Ani3D) и визуализации результатов расчётов (Paraview, Mayavi).

В части проведения расчетов и анализа результатов соискателем выполнено численное исследование волновых процессов в многослойных средах, моделирующих панель из полимерного композиционного материала при динамическом нагружении, получены области потенциальных разрушений различных типов, обусловленных распространением и взаимодействием волновых фронтов в материале. Проведено численное моделирование натурного эксперимента по динамическому нагружению элемента композитной обшивки и силового кессона крыла самолёта для двух постановок эксперимента -- удар по отдельному элементу обшивки и удар по элементу обшивки со стрингером. Для задачи со стрингером рассмотрены постановки с центральным и нецентральным ударом. Выполнен анализ областей концентрации напряжений, вызванных волновыми процессами в ходе соударения. Определены зоны потенциальных повреждений конструкции, обусловленные разными механизмами разрушения материала.

Дополнительно проведено численное исследование волновых процессов в покровах мозга при динамическом нагружении для многокомпонентной и упрощенных моделей. Выполнены расчеты задач о черепно-мозговой травме, о нагружении коленного сустава, об ударе по торсу в защитной конструкции.
