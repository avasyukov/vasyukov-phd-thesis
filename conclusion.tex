\section*{Заключение}

\subsection*{Основные результаты и выводы диссертации}

Разработан метод численного моделирования на неструктурированной сетке действия низкоскоростного удара на конструкцию сложной формы в трёхмерной постановке. Разработанный метод позволяет проводить моделирование волновых процессов в конструкции при динамическом внешнем воздействии с учетом взаимодействия волновых фронтов, влияния внешних и внутренних границ, различия реологических свойств слоёв. Особенностью метода является возможность выполнять расчёты с шагом $\tau > h / \lambda$ в трёхмерной постановке. Разработанный метод исследован на аппроксимацию и устойчивость. Проведено тестирование реализации метода.

Разработанный сеточно-характеристический метод реализован в виде параллельного вычислительного комплекса, позволяющего выполнять моделирование как на стандартном оборудовании, так и на современных высокопроизводительных вычислительных комплексах.

Выполнено исследование волновых процессов в многослойных средах различной структуры, моделирующих панель из полимерного композиционного материала. Исследование включает в себя как аналитическое, так и численное изучение процессов, протекающих в многослойной среде при динамическом нагружении. Получены поля скоростей и напряжений, области потенциальных разрушений различных типов, обусловленные распространением и взаимодействием волновых фронтов в материале.

Выполнено численное моделирование натурного эксперимента по динамическому нагружению элемента композитной обшивки и силового кессона крыла самолёта. Проведены расчеты для двух постановок эксперимента -- удар по отдельному элементу обшивки и удар по элементу обшивки со стрингером. Для задачи со стрингером рассмотрены постановки с центральным и нецентральным ударом. Проведен анализ причин разрушения композиционных авиационных материалов. Для всех постановок получены области концентрации напряжений, вызванные волновыми процессами в ходе соударения. Определены зоны потенциальных повреждений конструкции, обусловленные разными механизмами разрушения материала. Для элемента обшивки без стрингера размер разрушенной области составляет 50-60 мм, для элемента обшивки со стрингером 25-30 мм при центральном ударе и 20-25 мм при нецентральном ударе.

Получено, что наличие стрингера существенно разгружает элемент обшивки при динамическом воздействии и уменьшает размер потенциально повреждённых областей. Данный результат важен, так как при действии статической нагрузки наличие стрингера напротив вызывает концентрацию напряжений и приводит к разрушению при меньшей силе воздействия.

Разработанный численный метод применен для решения ряда задач биомеханики. Получены области потенциальных повреждений тканей организма человека в задачах о черепно-мозговой травме, о динамическом нагружении коленного сустава и об ударе по торсу в защитной конструкции.

Результаты численного моделирования действия низкоскоростного удара на конструкцию из полимерного композиционного материала могут быть использованы для экспериментальной проверки предложенных математических моделей и численного метода. Сформулированы критерии для сравнения численного и натурного эксперимента, учитывающие механические свойства распространённых полимерных матриц.

После экспериментальной проверки разработанные модели и методы могут быть использованы при создании методик и норм проверки прочностных характеристик композиционных материалов.

Разработанный параллельный программный комплекс может быть использован для моделирования динамического воздействия на комплексные силовые конструкции из композиционных материалов.

Полученные результаты по взаимодействию упругой волны с разрушенной областью конструкции могут быть использованы при разработке методов неразрушающего контроля состояния изделий из композиционных материалов.

Полученные результаты в части задач биомеханики могут быть использованы при разработке защитного снаряжения различных видов.
