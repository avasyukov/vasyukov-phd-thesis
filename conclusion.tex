\section*{Заключение}
\addcontentsline{toc}{section}{Заключение}

\subsection*{Основные результаты и выводы диссертации}

\begin{enumerate}

\item Разработана математическая модель панели из полимерного композиционного материала для задачи о низкоскоростном ударе по элементу композитной обшивки и силового кессона крыла самолёта. Модель может быть использована в том числе для более сложных инженерных конструкций, выполненных из композиционных материалов.

\item Разработан сеточно-характеристический метод на сетке из тетраэдров для моделирования удара по конструкции сложной формы в трёхмерной постановке. Разработанный метод позволяет проводить моделирование волновых процессов в конструкции при динамическом внешнем воздействии с учетом взаимодействия волновых фронтов, влияния внешних и внутренних границ, различия реологических свойств слоёв. Особенностью метода является возможность выполнять расчёты с шагом $\tau > h / \lambda$ (здесь $\tau$ -- шаг по времени, $h$ -- минимальное расстояние от узла сетки до соседних узлов, $\lambda$ -- максимальное по модулю собственное число упределяющей системы уравнений). Разработанный метод исследован на аппроксимацию и устойчивость. Проведено тестирование реализации метода.

\item Разработанный сеточно-характеристический метод реализован в виде параллельного вычислительного комплекса, позволяющего выполнять моделирование как на стандартном оборудовании, так и на современных высокопроизводительных вычислительных комплексах. Для явного выделения контактных границ при параллельном расчете разработан параллельный алгоритм детектора столкновений.

\item Реализованный параллельный вычислительный комплекс интегрирован с существующими программами задания геометрии объектов (Gmsh, Tetgen, Ani3D) и визуализации результатов расчётов (Paraview, Mayavi), являющимися стандартом де-факто среди инженеров-практиков.

\item Выполнено исследование волновых процессов в многослойных средах сложной структуры, моделирующих панель из полимерного композиционного материала. Исследование включает в себя как аналитическое, так и численное изучение процессов, протекающих в многослойной среде при динамическом нагружении (задачи об объёмных волнах, поверхностных волнах, волнах на контактной границе). Получены поля скоростей и напряжений, области потенциальных разрушений различных типов, обусловленные распространением и взаимодействием волновых фронтов в материале.

\item Выполнено численное моделирование натурного эксперимента по динамическому нагружению элемента композитной обшивки и силового кессона крыла самолёта. Проведены расчеты для двух постановок эксперимента -- удар по отдельному элементу обшивки и удар по элементу обшивки со стрингером. Для задачи со стрингером рассмотрены постановки с центральным и нецентральным ударом. Проведен анализ причин разрушения композиционных авиационных материалов. Для всех постановок получены области концентрации напряжений, вызванные волновыми процессами в ходе соударения. Определены зоны потенциальных повреждений конструкции, обусловленные разными механизмами разрушения материала. Для элемента обшивки без стрингера размер разрушенной области составляет 50-60 мм, для элемента обшивки со стрингером 25-30 мм при центральном ударе и 20-25 мм при нецентральном ударе. Получено, что наличие стрингера существенно разгружает элемент обшивки при динамическом воздействии и уменьшает размер потенциально повреждённых областей. Данный результат важен, так как при действии статической нагрузки наличие стрингера напротив вызывает концентрацию напряжений и приводит к разрушению при меньшей силе воздействия.

\item Дополнительно разработанные математические модели и численный метод были применены для решения ряда задач биомеханики. Получены области потенциальных повреждений тканей организма человека в задачах о черепно-мозговой травме, о динамическом нагружении коленного сустава и об ударе по торсу в защитной конструкции.

\end{enumerate}
