\begin{center}\section*{Заключение}\end{center}
\addcontentsline{toc}{section}{Заключение}

В результате выполнения НИР разработаны оригинальные методы численного моделирования действия низкоскоростного удара на конструкцию из полимерного композиционного материала (ПКМ) как в двумерной, так и в трёхмерной постановке. Разработанные методы позволяют проводить моделирование волновых процессов в композитной конструкции при динамическом внешнем воздействии с учетом взаимодействия волновых фронтов, влияния внешних и внутренних границ, различия реологических свойств слоёв.

Разработанные методы реализованы в виде параллельного вычислительного комплекса, позволяющего выполнять моделирование как на стандартном оборудовании, так и на современных высокопроизводительных вычислительных комплексах.

Выполнено исследование волновых процессов в многослойных средах различной структуры, моделирующих панель из полимерного композиционного материала. Исследование включает в себя как аналитическое, так и численное изучение процессов, протекающих в многослойной среде при динамическом нагружении. Получены области потенциальных разрушений различных типов, обусловленные распространением и взаимодействием волновых фронтов в материале.

Выполнено численное моделирование натурного эксперимента по динамическому нагружению элемента композитной обшивки и силового кессона крыла самолёта. Проведены расчеты для двух постановок эксперимента -- удар по отдельному элементу обшивки и удар по элементу обшивки со стрингером. Для задачи со стрингером рассмотрены постановки с центральным и нецентральным ударом. Для всех постановок получены области концентрации напряжений, вызванные волновыми процессами в ходе соударения. Определены зоны потенциальных повреждений конструкции, обусловленные разными механизмами разрушения материала. Для элемента обшивки без стрингера размер разрушенной области составляет 50-60 мм, для элемента обшивки со стрингером 25-30 мм при центральном ударе и 20-25 мм при нецентральном ударе.

Получено, что наличие стрингера существенно разгружает элемент обшивки при динамическом воздействии и уменьшает размер потенциально повреждённых областей. Данный результат важен, так как при действии статической нагрузки наличие стрингера напротив вызывает концентрацию напряжений и приводит к разрушению при меньшей силе воздействия.

Результаты численного моделирования, выполненного в рамках НИР, могут быть использованы для экспериментальной проверки предложенных математических моделей и численного метода. При сравнении численного и натурного эксперимента целесообразно обратить внимание на два критерия -- интегральная характеристика воздействия и максимальные растягивающие напряжения. Первый критерий описывает общее воздействие на конструкцию. Максимальные растягивающие напряжения принципиальны, так как по заявленным прочностным характеристикам монослоёв можно ожидать, что именно нагрузку такого типа данный композиционный материал будет выдерживать хуже всего.

После экспериментальной проверки разработанные модели и методы могут быть использованы при создании методик и норм проверки прочностных характеристик композиционных материалов. Разработанный программный комплекс может быть использован для моделирования динамического воздействия на силовые конструкции из композиционных материалов. Полученные результаты по взаимодействию упругой волны с разрушенной областью конструкции могут быть использованы при разработке методов неразрушающего контроля состояния изделий из композиционных материалов.

В развитие работы целесообразно рассмотреть протекание волновых процессов на следующих уровнях структуры композиционного материала -- отдельные монослои в составе субпакета, матрица и волокна внутри монослоя.

