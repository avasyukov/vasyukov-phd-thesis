\section{Моделирование композиционных материалов}

\subsection{Строение и структура композитов}

\subsection{Волновые процессы в сплошных средах разной структуры}

\subsubsection{Однородная среда}

В однородной изотропной среде существует только два типа волн: поперечные и продольные (доказано Пуассоном, доказательство приведено в \cite{amenadze}\todo{стр. 249}). При этом продольные волны при распространении не генерируют поперечных и наоборот. Вывод формул для распространения плоской продольной, плоской поперечной и сферической продольной волн также приведен в \cite{amenadze}\todo{стр. 250}.

В случае точечного источника волн математическая ситуация несколько усложняется. Вывод приведен в \cite{aki_richards}\todo{стр. 73}. Подробно свойства P-волны, SH-волны и SV-волны рассмотрены в \cite{aki_richards}\todo{стр. 77}. В наших задачах нас интересует, по сути, только дальняя зона.

В однородной среде с какой-либо границей продольные и поперечные волны распространяются независимо лишь до того момента, пока фронт не пересечет границу. Тогда образуются так называемые отраженные и преломленные волны обоих типов, так как обычно системе граничных условий нельзя удовлетворить, введя отраженную волну какого-либо одного типа \cite{amenadze}\todo{стр. 250}. Подробная математика отражения и преломления плоских волн на плоских границах в однородных средах рассмотрена в \cite{aki_richards}\todo{стр. 121}. Выводы и диаграммы по отражению от свободной границы приведены в \cite{aki_richards} \todo{стр. 132, 136, 137 (P и SV), 140 (SH)}. Выводы и диаграммы по отражению и преломлению на жесткой границе двух твердых тел также рассмотрены в \cite{aki_richards}\todo{стр. 142, 143 (P и SV)}.


\subsubsection{Упругое полупространство}

Рассмотрим упругое полупространство и задачу о его свободных колебаниях. На границе полупространства - условие свободной границы, то есть действующая сила равна нулю. Вывод уравнений для получающихся волн приведен в \cite{aki_richards}\todo{стр. 253} и \cite{viktorov}\todo{стр. 5}. Эти волны впервые были исследованы Рэлеем. Стоит заметить, что решение ищется при условиях однородности по третьей оси координат (плоская деформация) и затухания с глубиной ("поверхностная" волна). Любая фиксированная точка изучаемого тела или породы при этом будет двигаться по эллипсу. Волны Рэлея имеют большое значение для геологических исследований, так как представляют наибольшую опасность при землетрясениях. Энергия, которую эти волны несут, сконцентрирована у поверхности и рассеивается по поверхности, ее рассеивание происходит медленнее, чем в тех волнах, где энергия рассеивается по объему возмущенной области. Свойства волн Рэлея также описаны в \cite{aki_richards}\todo{стр. 156}, \cite{tischenko}. Заметим, что в \cite{aki_richards} произведен вывод формул для волн Релея с точки зрения отражения неоднородной по горизонтальной медленности P- и SV- волн.

В 1904 году Лэмб дал точное решение задачи, в которой источник действовал как импульс, приложенный к свободной границе твердого полупространства по нормали к ней \cite{lamb}. Однако теперь термин "задача Лэмба" относят к более общему случаю произвольного источника в среде с одной границей.

Математика отражения сферической волны от плоской свободной границы (цилиндрическая симметрия) подробно рассмотрена в \cite{aki_richards}\todo{стр. 206, краткие итоги анализа уравнений — стр. 211, свойства цилиндрической волны Рэлея — стр. 214}.

О способах генерации волн Рэлея написано в \cite{viktorov}\todo{стр. 12}.


\subsubsection{Два упругих полупространства}

При наличии двух упругих полупространств с различными свойствами, как показано в \cite{aki_richards}, стр. 156, по аналогии с волнами Рэлея возникают волны Стоунли. Такие волны всегда могут существовать на границе жидкости и твердого тела, но, при определенном соотношении свойств материалов, могут появляться и на границе двух твердых тел. Волны Стоунли, аналогично рэлеевским, не обладают дисперсией.

Горизонтальная плоская граница создает связь плоских волн P- и SV-, а волны SH- распространяются независимо от них \cite{aki_richards}, стр. 207).

Математика отражения и преломления сферической волны на плоской границе подробно рассмотрена в \cite{aki_richards}, стр. 192. Когда сферическая волна взаимодействует с плоской границей между двумя различными полупространствами, образующуюся систему волн можно естественно разделить на три основных группы: 1) волны, естественно отраженные от границы или преломленные на ней; 2) головные волны; 3) волны типа Рэлея и Стоунли (\cite{aki_richards}, стр. 186). Заметим, что головные волны — это не прямые волны, которые появляются на приемнике при распространении непосредственно от источника, а отдельный тип волн с той же скоростью распространения, но с другими свойствами (\cite{aki_richards}, стр. 203).


\subsubsection{Один слой и упругое полупространство}

Рассмотрим упругий слой постоянной толщины, лежащий на упругом полупространстве из другого материала (с большей скоростью распространения поперечных волн). Ищем поперечную волну, которая будет распространяться вдоль границы раздела как в слое, так и в полупространстве. Подробный вывод уравнений - \cite{amenadze}, стр. 256. Такие волны называется волнами Лява. Аналогично рэлеевским волнам, энергия волн Лява концентрируется вблизи границы раздела двух сред, поэтому может наблюдаться на значительном удалении от эпицентра возмущения. В отличие от релеевских волн, волны Лява имеют дисперсию (фазовая скорость зависит от частоты).


\subsubsection{Один слой}

Случай тонкой пластинки рассмотрен в \cite{amenadze}, стр. 259. 

Волны Лэмба (волны в пластинке конечной толщины) подробно рассмотрены в \cite{viktorov}, стр. 78 (математика), стр. 87 (свойства), стр. 96 (возбуждение). На стр. 107 приведено сравнение волн Лэмба с волнами Рэлея. Делается вывод, что при стремлении толщины пластинки к бесконечности волны Лэмба переходят в волну Рэлея. При конечной толщине можно говорить о квазирэлеевской волне. Она распространяется вдоль поверхности, к которой прилагалось начальное возмущение, и на некотором расстоянии, зависящем от толщины пластинки, переходит на противоположную поверхность. 


\subsubsection{Слойка}

Случай жидкого слоя в твердом слоистом полупространстве рассмотрен в \cite{aki_richards}, стр. 266.


\subsubsection{Объемные волны}

Продольные волны (они же объемные волны) — волны, в которых направление распространения совпадает с мнговенной скоростью частиц. Они наблюдаются при возмущении в однородной среде (или при достаточной удаленности от границ).

Поперечные волны (они же сдвиговые волны) — волны, в которых направление распространения первендикулярно мнговенной скорости частиц. Они также наблюдаются в однородной среде. Условное разделение поперечных волн на SH и SV происходит при появлении плоской границы. SH-волнами называют поперечные волны, мгновенная скорость частиц в которой параллельна этой плоской границе. Они появляются только в трехмерном случае, в двухмерном случае (а также квазидвухмерном или цилиндрически симметричном) таких волн нет. Соответственно, SV-волны — это любые другие.

Продольные и поперечные волны могут иметь различную форму фронта. В аналитических исследованиях чаще всего говорят о плоском и сферическом фронтах, как самых простых для математического анализа. В задаче о точечном источнике обычно фигурирует сферическая форма фронта.

Продольные и сдвиговые волны не зависят друг от друга в однородном пространстве. При наличии границ появляется связь между P- и SV- волнами, так как при отражении и преломлении они друг друга порождают. SH-волны распространяются независимо от P- и SV-.

Других волн в однородном пространстве не наблюдается, что доказано Пуассоном.


\subsubsection{Поверхностные волны}
Волны Рэлея появляются в упругом полупространстве при наличии свободной границы. В случае более сложной геометрии говорить о наличии волн Рэлея не вполне корректно, однако если полупространство является однородным на расстоянии, большем длины волны (достаточно широкий верхний слой), то данный термин вполне применим, а сравнение с аналитическим выводом правомерно. Волны Рэлея появляются как в двухмерной постановке, так и в трехмерной. Достаточно просто они генерируются при обычной постановке задачи Лэмба (удар по свободной поверхности) и при отражении сферического фронта продольной волны. В трехмерной постановке фронт волны Рэлея является цилиндрическим. В волне Рэлея частицы движутся по эллипсу.

Волны Стоунли полностью аналогичны волнам Рэлея, но появляются при наличии плоской границы между двумя упругими полупространствами. Аналогично, об их наличии мы можем говорить только в том случае, если толщина слоя достаточно велика, как в двухмерном, так и в трехмерном случае.

Волны Лява наблюдаются только в трехмерном случае и возникают только на границе между упругим полупространством и лежащем на нём слое с верхней свободной границей. Движение частиц параллельно плоскости границы и перепендикулярно направлению распространения. Эти волны легко отличить от волн Стоунли и Рэлея, так как мгновенное направление скоростей частиц в них перпендикулярны аналогичным у последних.

Головные волны распространяются к приемнику от источника по пути, включающему скольжение вдоль границы со скоростью объемной волны.

Волны Лэмба наблюдаются в случае пластинки конечной толщины. Они бывают симметричными и антисимметричными, причем количество их конечно и зависит от толщины пластинки. При достаточной толщине интерференция волн Лэмба нулевого порядка дает квазирэлеевскую волну, которая на некотором расстоянии от точки начального возмущения переходит на противоположную поверхность, а при бесконечной толщине стремится к рэлеевской. 

Других волн в случае рассмотренных геометрий (полупространство, два полупространства, слой на полупространстве, один слой) не наблюдается.


\subsubsection{Постановка задачи для многослойной конструкции}

Можно сделать вывод, что при рассмотрении многослойной конструкции с достаточно широкими (в сравнении с длинами волн) слоями, мы можем говорить о наличии волн Рэлея в верхнем слое и волн Стоунли в остальных слоях. При асимметричном начальном возмущении мы можем получить волну, которую будет достаточно корректно назвать волной Лява. Обычно для ее получения в численном расчете задается достаточно сложная геометрия начального импульса, но она должна формироваться при любом неоднородном и несимметричном начальном возмущении в постановке, аналогичной постановке задачи Лэмба.

Волны Лэмба также могут нас интересовать: в случае однородной пластинки мы можем сравнить с аналитикой расстояние, на котором происходит переход квазирэлеевской волны (интерференции волн Лэмба нулевого порядка) на противоположную поверхность. Возможно, аналогичный эффект перехода будет наблюдаться и при несвободной нижней границе, т.е. в многослойной конструкции.

Также заметим, что для получения поверхностных волн необходим достаточно долгий расчет: они формируются не сразу. Однако со временем они не только демонстрируют характерные и легко отличимые эллиптические фронты, но и выделяются по масштабу, так как затухают гораздо медленнее объемных волн. 

Для сравнения нашего метода с аналитикой лучше брать самые простые постановки. Упругое полупространство с плоской свободной границей, два упругих полупространства с полным слипанием между ними, однородную плоскую пластинку. Плоский (под различными углами) или сферический фронты начальных возмущений. Количественно сравнивать с аналитикой при этом можно амплитуды отраженной и преломленной волны (плоский фронт под различными углами), скорость и амплитуду волн Рэлея и Стоунли (сферический фронт), расстояние перехода и амплитуду квазирэлеевской волны в случае однородной пластинки (интерференция волн Лэмба). Аналитические выражения достаточно взять из \cite{aki_richards} и \cite{viktorov}. Учитывая сложность математики, приведенной для простых случаев, многослойную постановку в аналитикой сравнивать в ближайшее время не получится. Однако при достаточной ширине слоев (больше длин поверхностных волн) мы можем говорить об аналогии полученных волн с известными нам волнами Рэлея и Стоунли.


\subsection{Методы и подходы к моделированию композитов}

