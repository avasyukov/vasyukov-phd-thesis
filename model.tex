\section{Математическая модель}

\subsection{Уравнения механики твёрдого тела}
Для математического моделирования волновых процессов в деформируемом твёрдом
теле используется система динамических уравнений \cite{novatsky,sedov} в виде
\begin{eqnarray}
\label{initial_equations}
\rho\dot{v}_i=\nabla_j\sigma_{ij}+f_i & \textrm{(уравнения движения)}\nonumber\\
\sigma_{ij}=q_{ijkl}\dot{\varepsilon}_{kl}+F_{ij} & \textrm{(реологические
соотношения).}
\end{eqnarray}

Здесь $\rho$ – плотность среды, $v_i$ – компоненты скорости смещения,
$\sigma_{ij}$, $\varepsilon_{ij}$ -- компоненты тензоров напряжений и деформаций,
$\nabla_j$ – ковариантная производная по $j$-й координате, $f_i$ – массовые
силы, действующие на единицу объёма, $F_{ij}$ -- правая часть, используемая, например, для описания диссипации в моделях с учётом вязкости.

В случае малых деформаций тензор скоростей деформаций $e_{ij}=\dot{\varepsilon}_{ij}$ 
выражается через компоненты скорости смещения линейным образом:
\begin{equation}
e_{ij}=\frac{1}{2}(\nabla_j v_i+\nabla_i v_j).
\end{equation}

Вид компонент тензора 4-го порядка $q_{ijkl}$ и правой части $F_{ij}$ определяется реологией среды.

Для замыкания системы уравнений \ref{initial_equations} её необходимо дополнить
уравнением состояния, определяющим зависимость плотности от напряжений:
$$\rho=\rho_0e^{\frac{p}{K}},$$
где $p=-\frac{1}{3}\sum\sigma_{kk}$ -- давление, $K=\lambda+\frac{2}{3}\mu$ --
коэффициент всестороннего сжатия, $\lambda$ и $\mu$ -- параметры Ламе.

Параметры Ламе зависят от материала и связаны с модулем продольной упругости и коэффициентом Пуассона следующим образом:
\begin{eqnarray}
\label{lame_parameters}
\lambda=\frac{E\nu}{(1+\nu)(1-2\nu)}
\nonumber\\
\mu=G=\frac{E}{2(1+\nu)}
\end{eqnarray}
Здесь $E$ -- модуль продольной упругости, $\nu$ -- коэффициент Пуассона, $G$ -- модуль сдвига.

В простейшем случае линейной упругости $q_{ijkl}=\lambda\delta_{ij}\delta_{kl}+\mu(\delta_{ik}\delta_{jl}+\delta_{il}\delta_{jk})$ и $F_{ij}=0$. Тогда в приближении малых деформаций и в отсутствии внешних сил в трехмерном пространстве и декартовых координатах уравнения \ref{initial_equations} принимают вид

\begin{eqnarray}
\label{simple_equations}
\frac{\partial{v_x}}{\partial{t}}=\frac{1}{\rho}(\frac{\partial{\sigma_{xx}}}{\partial{x}}+\frac{\partial{\sigma_{xy}}}{\partial{y}}+\frac{\partial{\sigma_{xz}}}{\partial{z}})
\nonumber\\
\frac{\partial{v_y}}{\partial{t}}=\frac{1}{\rho}(\frac{\partial{\sigma_{xy}}}{\partial{x}}+\frac{\partial{\sigma_{yy}}}{\partial{y}}+\frac{\partial{\sigma_{yz}}}{\partial{z}})
\nonumber\\
\frac{\partial{v_z}}{\partial{t}}=\frac{1}{\rho}(\frac{\partial{\sigma_{xz}}}{\partial{x}}+\frac{\partial{\sigma_{yz}}}{\partial{y}}+\frac{\partial{\sigma_{zz}}}{\partial{z}})
\nonumber\\
\frac{\partial{\sigma_{xx}}}{\partial{t}}=(\lambda+2\mu)\frac{\partial{v_x}}{\partial{x}}+\lambda\frac{\partial{v_y}}{\partial{y}}+\lambda\frac{\partial{v_z}}{\partial{z}}
\nonumber\\
\frac{\partial{\sigma_{xy}}}{\partial{t}}=\mu(\frac{\partial{v_x}}{\partial{y}}+\frac{\partial{v_y}}{\partial{x}})
\nonumber\\
\frac{\partial{\sigma_{xz}}}{\partial{t}}=\mu(\frac{\partial{v_x}}{\partial{z}}+\frac{\partial{v_z}}{\partial{x}})
\nonumber\\
\frac{\partial{\sigma_{yy}}}{\partial{t}}=\lambda\frac{\partial{v_x}}{\partial{x}}+(\lambda+2\mu)\frac{\partial{v_y}}{\partial{y}}+\lambda\frac{\partial{v_z}}{\partial{z}}
\nonumber\\
\frac{\partial{\sigma_{yz}}}{\partial{t}}=\mu(\frac{\partial{v_z}}{\partial{y}}+\frac{\partial{v_y}}{\partial{z}})
\nonumber\\
\frac{\partial{\sigma_{zz}}}{\partial{t}}=\lambda\frac{\partial{v_x}}{\partial{x}}+\lambda\frac{\partial{v_y}}{\partial{y}}+(\lambda+2\mu)\frac{\partial{v_z}}{\partial{z}}
\end{eqnarray}

Очевидно, что уравнения \ref{simple_equations} можно переписать в матричной форме:
\begin{equation}
\label{simple_matrix_equation}
\frac{\partial\vec{u}}{\partial{t}}+\mathbf{A}_x\frac{\partial\vec{u}}{\partial{x}}+
\mathbf{A}_y\frac{\partial\vec{u}}{\partial{y}}+
\mathbf{A}_z\frac{\partial\vec{u}}{\partial{z}}=0.
\end{equation}
Здесь
$\vec{u}=\{v_x,v_y,v_z,\sigma_{xx},\sigma_{yy},\sigma_{zz},\sigma_{xy},\sigma_{xz},\sigma_{yz}\}^T$
-- вектор искомых функций, $x,y,z$ --  независимые пространственные переменные, $t$ -- время.

Аналогично можно записать более общую систему \ref{initial_equations} в виде:
\begin{equation}
\label{matrix_equation}
\frac{\partial\vec{u}}{\partial{t}}+\mathbf{A}_x\frac{\partial\vec{u}}{\partial{x}}+
\mathbf{A}_y\frac{\partial\vec{u}}{\partial{y}}+
\mathbf{A}_z\frac{\partial\vec{u}}{\partial{z}}=\vec{f}.
\end{equation}
Здесь $\vec{f}$ -- вектор правых частей, размерность которого равна размерности исходной системы, а выражения для компонентов зависят от реологии среды. Точный вид матриц $\mathbf{A}_x$, $\mathbf{A}_y$, $\mathbf{A}_z$ также зависит от реологии среды.

\clearpage
\newpage

\subsection{Приближение линейно упругого тела}

\subsubsection{Реологические соотношения для линейно упругого тела}

Для линейно упругого тела тензор $q_{ijkl}$ и правая часть $F_{ij}$ в \ref{initial_equations} принимают следующий вид:
\begin{eqnarray}
\label{tensor_qijkl_elastic}
q_{ijkl}=\lambda\delta_{ij}\delta_{kl}+\mu(\delta_{ik}\delta_{jl}+\delta_{il}
\delta_{jk}),\nonumber\\
F_{ij}=0.
\end{eqnarray}
В этом соотношении $\lambda$ и $\mu$ -- параметры Ламе, $\delta_{ij}$ -- символ Кронекера.

\subsubsection{Матричная форма уравнений для линейно упругого тела}
\label{elastic_matrixes}

Для линейно упругого тела матрицы $\mathbf{A}_x$, $\mathbf{A}_y$, $\mathbf{A}_z$ в \ref{matrix_equation} принимают следующий вид.

\begin{displaymath}
\mathbf{A}_x =
\left( \begin{array}{cccccccccccc}
0 & 0 & 0 & -\frac 1 \rho & 0 & 0 & 0 & 0 & 0 \\ 
0 & 0 & 0 & 0 & -\frac 1 \rho & 0 & 0 & 0 & 0 \\ 
0 & 0 & 0 & 0 & 0 & -\frac 1 \rho & 0 & 0 & 0 \\ 
-(\lambda+2\mu) & 0 & 0 & 0 & 0 & 0 & 0 & 0 & 0 \\ 
0 & -\mu & 0 & 0 & 0 & 0 & 0 & 0 & 0 \\ 
0 & 0 & -\mu & 0 & 0 & 0 & 0 & 0 & 0 \\ 
-\lambda & 0 & 0 & 0 & 0 & 0 & 0 & 0 & 0 \\ 
0 & 0 & 0 & 0 & 0 & 0 & 0 & 0 & 0 \\ 
-\lambda & 0 & 0 & 0 & 0 & 0 & 0 & 0 & 0  
\end{array} \right),
\end{displaymath} 
\begin{displaymath}
\mathbf{A}_y =
\left( \begin{array}{cccccccccccc}
0 & 0 & 0 & 0 & -\frac 1 \rho & 0 & 0 & 0 & 0 \\ 
0 & 0 & 0 & 0 & 0 & 0 & -\frac 1 \rho & 0 & 0 \\ 
0 & 0 & 0 & 0 & 0 & 0 & 0 & -\frac 1 \rho & 0 \\ 
0 & -\lambda & 0 & 0 & 0 & 0 & 0 & 0 & 0 \\ 
-\mu & 0 & 0 & 0 & 0 & 0 & 0 & 0 & 0 \\ 
0 & 0 & 0 & 0 & 0 & 0 & 0 & 0 & 0 \\ 
0 & -(\lambda+2\mu) & 0 & 0 & 0 & 0 & 0 & 0 & 0 \\ 
0 & 0 & -\mu & 0 & 0 & 0 & 0 & 0 & 0 \\ 
0 & -\lambda & 0 & 0 & 0 & 0 & 0 & 0 & 0  
\end{array} \right),
\end{displaymath}
\begin{displaymath}
\mathbf{A}_z =
\left( \begin{array}{cccccccccccc}
0 & 0 & 0 & 0 & 0 & -\frac 1 \rho & 0 & 0 & 0 \\ 
0 & 0 & 0 & 0 & 0 & 0 & 0 & -\frac 1 \rho & 0 \\ 
0 & 0 & 0 & 0 & 0 & 0 & 0 & 0 & -\frac 1 \rho \\ 
0 & 0 & -\lambda & 0 & 0 & 0 & 0 & 0 & 0 \\ 
0 & 0 & 0 & 0 & 0 & 0 & 0 & 0 & 0 \\ 
-\mu & 0 & 0 & 0 & 0 & 0 & 0 & 0 & 0 \\ 
0 & 0 & -\lambda & 0 & 0 & 0 & 0 & 0 & 0 \\ 
0 & -\mu & 0 & 0 & 0 & 0 & 0 & 0 & 0 \\ 
0 & 0 & -(\lambda+2\mu) & 0 & 0 & 0 & 0 & 0 & 0  
\end{array} \right).
\end{displaymath}

\todo{Добавить полное исследование системы и явный вид матриц $\Lambda$ и $\Omega$.}

\clearpage
\newpage

\subsection{Приближение упруго-пластического тела}

\subsubsection{Реологические соотношения для упруго-пластического тела (модель Прандтля-Рейсса с условием текучести Мизеса)}

Для упруго-пластического тела используется модель Прандтля-Рейсса с условием текучести Мизеса. Тогда тензор $q_{ijkl}$ и правая часть $F_{ij}$ в \ref{initial_equations} имеют более сложный вид:
\begin{eqnarray}
\label{tensor_qijkl_plastic}
q_{ijkl}=\lambda\delta_{ij}\delta_{kl}+\mu(\delta_{ik}\delta_{jl}+\delta_{il}\delta_{jk})-\frac{I\mu\sigma_{ij}\sigma_{kl}}{K^2},
\nonumber\\
F_{ij}=0.
\end{eqnarray}
В этом соотношении $\lambda$ и $\mu$ -- параметры Ламе, $K$ -- предел текучести на сдвиг, $\sigma_{ij}$ -- компоненты тензора напряжений, $\delta_{ij}$ -- символ Кронекера, $I$ -- параметр модели, который определяется следующим образом:

\begin{equation}
\label{I_parameter_plastic}
I=\begin{cases}
0, & \text{если $S = \sigma_{xx}^2+\sigma_{yy}^2+\sigma_{zz}^2+2\sigma_{xy}^2+2\sigma_{xz}^2+2\sigma_{yz}^2 < 2K^2$}\\
1, & \text{если $S \ge 2K^2$}.
\end{cases}
\end{equation}

\subsubsection{Матричная форма уравнений для модели Прандтля-Рейсса}
\label{plastic_matrixes}

Для линейно упругого тела матрицы $\mathbf{A}_x$, $\mathbf{A}_y$, $\mathbf{A}_z$ в \ref{matrix_equation} имеют существенно более сложный вид, так как компоненты тензора $q_{ijkl}$ зависят от компонентов тензора $\sigma$. Значения $\sigma_{ij}$ в общем случае различны в каждой точке пространства в каждый момент времени. Это приводит к тому, что невозможно упростить вид матриц аналитически и получить их покомпонентную запись в терминах $\lambda, \mu, \rho$, как это было сделано для линейно упругого тела. Значения ненулевых элементов каждой матрицы необходимо вычислять в каждой точке пространства в каждый момент времени в соответствии с \ref{tensor_qijkl_plastic} и \ref{I_parameter_plastic}, используя текущие значения $\sigma_{ij}$ в данной точке.

\begin{displaymath}
\mathbf{A}_x =
\left( \begin{array}{cccccccccccc}
0 & 0 & 0 & -\frac 1 \rho & 0 & 0 & 0 & 0 & 0 \\ 
0 & 0 & 0 & 0 & -\frac 1 \rho & 0 & 0 & 0 & 0 \\ 
0 & 0 & 0 & 0 & 0 & -\frac 1 \rho & 0 & 0 & 0 \\ 
-q_{1111} & -\frac{q_{1112}+q_{1121}}{2} & -\frac{q_{1113}+q_{1131}}{2} & 0 & 0 & 0 & 0 & 0 & 0 \\ 
-q_{1211} & -\frac{q_{1212}+q_{1221}}{2} & -\frac{q_{1213}+q_{1231}}{2} & 0 & 0 & 0 & 0 & 0 & 0 \\ 
-q_{1311} & -\frac{q_{1312}+q_{1321}}{2} & -\frac{q_{1313}+q_{1331}}{2} & 0 & 0 & 0 & 0 & 0 & 0 \\ 
-q_{2211} & -\frac{q_{2212}+q_{2221}}{2} & -\frac{q_{2213}+q_{2231}}{2} & 0 & 0 & 0 & 0 & 0 & 0 \\ 
-q_{2311} & -\frac{q_{2312}+q_{2321}}{2} & -\frac{q_{2313}+q_{2331}}{2} & 0 & 0 & 0 & 0 & 0 & 0 \\ 
-q_{3311} & -\frac{q_{3312}+q_{3321}}{2} & -\frac{q_{3313}+q_{3331}}{2} & 0 & 0 & 0 & 0 & 0 & 0  
\end{array} \right),
\end{displaymath} 
\begin{displaymath}
\mathbf{A}_y =
\left( \begin{array}{cccccccccccc}
0 & 0 & 0 & 0 & -\frac 1 \rho & 0 & 0 & 0 & 0 \\ 
0 & 0 & 0 & 0 & 0 & 0 & -\frac 1 \rho & 0 & 0 \\ 
0 & 0 & 0 & 0 & 0 & 0 & 0 & -\frac 1 \rho & 0 \\ 
-\frac{q_{1112}+q_{1121}}{2} & -q_{1122} & -\frac{q_{1123}+q_{1132}}{2} & 0 & 0 & 0 & 0 & 0 & 0 \\ 
-\frac{q_{1212}+q_{1221}}{2} & -q_{1222} & -\frac{q_{1223}+q_{1232}}{2} & 0 & 0 & 0 & 0 & 0 & 0 \\ 
-\frac{q_{1312}+q_{1321}}{2} & -q_{1322} & -\frac{q_{1323}+q_{1332}}{2} & 0 & 0 & 0 & 0 & 0 & 0 \\ 
-\frac{q_{2212}+q_{2221}}{2} & -q_{2222} & -\frac{q_{2223}+q_{2232}}{2} & 0 & 0 & 0 & 0 & 0 & 0 \\ 
-\frac{q_{2312}+q_{2321}}{2} & -q_{2322} & -\frac{q_{2323}+q_{2332}}{2} & 0 & 0 & 0 & 0 & 0 & 0 \\ 
-\frac{q_{3312}+q_{3321}}{2} & -q_{3322} & -\frac{q_{3323}+q_{3332}}{2} & 0 & 0 & 0 & 0 & 0 & 0  
\end{array} \right),
\end{displaymath}
\begin{displaymath}
\mathbf{A}_z =
\left( \begin{array}{cccccccccccc}
0 & 0 & 0 & 0 & 0 & -\frac 1 \rho & 0 & 0 & 0 \\ 
0 & 0 & 0 & 0 & 0 & 0 & 0 & -\frac 1 \rho & 0 \\ 
0 & 0 & 0 & 0 & 0 & 0 & 0 & 0 & -\frac 1 \rho \\ 
-\frac{q_{1113}+q_{1131}}{2} & -\frac{q_{1123}+q_{1132}}{2} & -q_{1133} & 0 & 0 & 0 & 0 & 0 & 0 \\ 
-\frac{q_{1213}+q_{1231}}{2} & -\frac{q_{1223}+q_{1232}}{2} & -q_{1233} & 0 & 0 & 0 & 0 & 0 & 0 \\ 
-\frac{q_{1313}+q_{1331}}{2} & -\frac{q_{1323}+q_{1332}}{2} & -q_{1333} & 0 & 0 & 0 & 0 & 0 & 0 \\ 
-\frac{q_{2213}+q_{2231}}{2} & -\frac{q_{2223}+q_{2232}}{2} & -q_{2233} & 0 & 0 & 0 & 0 & 0 & 0 \\ 
-\frac{q_{2313}+q_{2331}}{2} & -\frac{q_{2323}+q_{2332}}{2} & -q_{2333} & 0 & 0 & 0 & 0 & 0 & 0 \\ 
-\frac{q_{3313}+q_{3331}}{2} & -\frac{q_{3323}+q_{3332}}{2} & -q_{3333} & 0 & 0 & 0 & 0 & 0 & 0  
\end{array} \right).
\end{displaymath}

\todo{Добавить про вид матриц $\Lambda$ и $\Omega$.}

\clearpage
\newpage

\subsection{Приближение вязко-упругого тела}

\subsubsection{Реологические соотношения Максвелла для вязко-упругого тела}

Для вязко-упругого тела при использовании модели Максвелла тензор $q_{ijkl}$ и правая часть $F_{ij}$ в \ref{initial_equations} принимают следующий вид:
\begin{eqnarray}
\label{tensor_qijkl_viscosity}
q_{ijkl}=\lambda\delta_{ij}\delta_{kl}+\mu(\delta_{ik}\delta_{jl}+\delta_{il}
\delta_{jk}),\nonumber\\
F_{ij}=-\frac{\sigma_{ij}}{\tau_0}.
\end{eqnarray}
В этом соотношении $\lambda$ и $\mu$ -- параметры Ламе, $\delta_{ij}$ -- символ Кронекера, $\tau_0$ -- время релаксации.

\subsubsection{Матричная форма уравнений для модели Максвелла}
\label{viscosity_matrixes}

Для вязко-упругого тела матрицы $\mathbf{A}_x$, $\mathbf{A}_y$, $\mathbf{A}_z$ в \ref{matrix_equation} принимают вид, полностью совпадающий со случаем линейно упругого тела (см.раздел\ref{elastic_matrixes}). Отличия только в правой части уравнений, в которой для вязко-упругого тела возникают члены, отвечающие за диссипацию.


\subsubsection{Наследственная вязко-упругая среда (модель Работнова)}

При использовании модели Работнова наследственной вязко-упругой среды тензор $q_{ijkl}$ и правая часть $F_{ij}$ в \ref{initial_equations} принимают следующий вид:
\begin{eqnarray}
\label{tensor_qijkl_rabotnov}
q_{ijkl}=\lambda\delta_{ij}\delta_{kl}+\mu(\delta_{ik}\delta_{jl}+\delta_{il}
\delta_{jk}),\nonumber\\
F_{ij}=-(L^*+M^*)\sigma_{ij}.
\end{eqnarray}
В этом соотношении $\lambda$ и $\mu$ -- параметры Ламе, $\delta_{ij}$ -- символ Кронекера. Правая часть имеет достаточно сложный вид и описывает поведение материала с учетом истории его нагружения. Оператор $L^*$ учитывает вязкие эффекты, оператор $M^*$ -- накопление повреждений. Вид операторов:
\begin{eqnarray}
\label{right_hand_rabotnov}
L^*\sigma_{ij} = \int_0^T{K_L(1-\tau)\sigma_{ij}d\tau}, \nonumber\\
M^*\sigma_{ij} = \int_0^T{K_M(1-\tau)\sigma_{ij}d\tau}, \nonumber\\
K_L(1-\tau)=\rho(1-\tau)^{-\alpha}, \nonumber\\
K_M(1-\tau)=m(1-\tau)^{-\alpha}.
\end{eqnarray}
Здесь $K_L(1-\tau)$ и $K_M(1-\tau)$ -- разностные ядра Абеля.

При использовании модели Работнова уравнения перестают быть локальными, для поиска решения на временном слое $n+1$ требуется интегрирование, вообще говоря, за все время жизни среды.

\subsubsection{Матричная форма уравнений для модели Работнова}
\label{rabotnov_matrixes}

Для модели Работнова матрицы $\mathbf{A}_x$, $\mathbf{A}_y$, $\mathbf{A}_z$ в \ref{matrix_equation} принимают вид, полностью совпадающий со случаем линейно упругого тела (см.раздел\ref{elastic_matrixes}). Отличия только в правой части уравнений, в которой возникают члены, отвечающие за вязкие эффекты и накопление повреждений.

\clearpage
\newpage

\subsection{Приближение вязко-упруго-пластического тела}

\subsubsection{Реологические соотношения для вязко-упруго-пластического тела (модель Кукуджанова)}

Для вязко-упруго-пластического тела при использовании модели Кукуджанова тензор $q_{ijkl}$ и правая часть $F_{ij}$ в \ref{initial_equations} принимают следующий вид:
\begin{eqnarray}
\label{tensor_qijkl_kukudzhanov}
q_{ijkl}=\lambda\delta_{ij}\delta_{kl}+\mu(\delta_{ik}\delta_{jl}+\delta_{il}
\delta_{jk}),\nonumber\\
F_{ij}=-\frac{2\mu}{\tau_0\sigma_{kl}\sigma_{kl}} F(\sigma_{kl},\sigma_{kl},K)\sigma_{ij},\nonumber\\
F(\sigma_{kl},\sigma_{kl},K) = \frac{(\sigma_{kl},\sigma_{kl})^{1/2}-(2K)^{1/2}}{\tau_0(\sigma_{kl},\sigma_{kl})^{1/2}}
\end{eqnarray}
В этом соотношении $\lambda$ и $\mu$ -- параметры Ламе, $\delta_{ij}$ -- символ Кронекера, $\tau_0$ -- время релаксации, $K$ -- предел текучести на сдвиг.

\subsubsection{Матричная форма уравнений для модели Кукуджанова}
\label{kukudzhanov_matrixes}

Для вязко-упруго-пластического тела матрицы $\mathbf{A}_x$, $\mathbf{A}_y$, $\mathbf{A}_z$ в \ref{matrix_equation} принимают вид, полностью совпадающий со случаем линейно упругого тела (см.раздел\ref{elastic_matrixes}). Отличия только в правой части уравнений.


\clearpage
\newpage

\subsection{Преобразование уравнений при смене базиса}

При практической реализации численного метода для решения системы \ref{matrix_equation} достаточно часто возникает необходимость смены базиса. Это может быть связано, например, с необходимостью случайного выбора направления координатных осей. Случайный выбор базиса используется для того, чтобы исключить наличие выделенных направлений, которые приводят к анизотропии численного решения. Еще одной частой причиной смены базиса является расчет областей с границей сложной формы. В этом случае при расчете граничных точек метод может требовать определенной ориентации осей базиса относительно внешней нормали в данной точке границы.

Исследуем, как преобразуются уравнения при смене базиса. Для этого рассмотрим исходный базис $\vec x = (x, y, z)^T$ и новый базис $\vec \xi = (\xi, \eta, \zeta)^T$. Пусть $\mathbf G$ - матрица, столбцами которой являются координаты векторов нового базиса $\vec \xi$ в старом базисе $\vec x$.

Тогда координаты точек, их скорости, а также компоненты тензора напряжений при переходе между базисами преобразуются следующим образом:
\begin{eqnarray}
\label{basis_change}
\vec r_x = \mathbf G \vec r_{\xi},
\nonumber\\
\vec v_x = \mathbf G \vec v_{\xi},
\nonumber\\
\vec \sigma_x = \mathbf G \sigma_{\xi} \mathbf G,
\nonumber\\
\vec r_{\xi} = {\mathbf G}^{-1} \vec r_x,
\nonumber\\
\vec v_{\xi} = {\mathbf G}^{-1} \vec v_x,
\nonumber\\
\vec \sigma_{\xi} = {\mathbf G}^{-1} \sigma_x {\mathbf G}^{-1}.
\end{eqnarray}

Здесь индекс $x$ относится к величинам, заданным в старом базисе, а индекс $\xi$ -- в новом.

При программной реализации того или иного метода решения системы \ref{matrix_equation} удобно задавать входные параметры и получать результаты в некоторой фиксированной декартовой системе координат $\vec x$, так как такой формат является естественным для предметной области. Что принципиально с точки зрения реализации метода -- уравнения \ref{matrix_equation} записаны в форме, независимой от выбора той или иной системы координат. Поэтому можно хранить состояние среды (компоненты скорости и напряжения) в фиксированном базисе $\vec x$, а вычисления вести в некотором локальном базисе $\vec \xi$, удобном с точки зрения алгоритма численного метода и локальной конфигурации узлов расчётной сетки. В этом случае необходимо лишь специально указать, какие операторы подставлять вместо компонент градиента.

Переход к производным по направлениям нового базиса в \ref{matrix_equation} приводит к следующим преобразованиям:
\begin{eqnarray}
\nonumber
\vec{f} = \frac{\partial\vec{u}}{\partial{t}} + 
\mathbf{A}_x\frac{\partial\vec{u}}{\partial{x}} + 
\mathbf{A}_y\frac{\partial\vec{u}}{\partial{y}} + 
\mathbf{A}_z\frac{\partial\vec{u}}{\partial{z}} =
\nonumber\\
\frac{\partial\vec{u}}{\partial{t}} + 
\mathbf{A}_x (\frac{\partial\vec{u}}{\partial{\xi}}\frac{\partial{\xi}}{\partial{x}} + 
\frac{\partial\vec{u}}{\partial{\eta}}\frac{\partial{\eta}}{\partial{x}} + 
\frac{\partial\vec{u}}{\partial{\zeta}}\frac{\partial{\zeta}}{\partial{x}} ) + 
\nonumber\\
\mathbf{A}_y (\frac{\partial\vec{u}}{\partial{\xi}}\frac{\partial{\xi}}{\partial{y}} + 
\frac{\partial\vec{u}}{\partial{\eta}}\frac{\partial{\eta}}{\partial{y}} + 
\frac{\partial\vec{u}}{\partial{\zeta}}\frac{\partial{\zeta}}{\partial{y}} ) + 
\nonumber\\
\mathbf{A}_z (\frac{\partial\vec{u}}{\partial{\xi}}\frac{\partial{\xi}}{\partial{z}} + 
\frac{\partial\vec{u}}{\partial{\eta}}\frac{\partial{\eta}}{\partial{z}} + 
\frac{\partial\vec{u}}{\partial{\zeta}}\frac{\partial{\zeta}}{\partial{z}} ) = 
\nonumber\\
\frac{\partial\vec{u}}{\partial{t}} + 
( \frac{\partial{\xi}}{\partial{x}} \mathbf{A}_x  + 
\frac{\partial{\xi}}{\partial{y}} \mathbf{A}_y + 
\frac{\partial{\xi}}{\partial{z}} \mathbf{A}_z ) \frac{\partial\vec{u}}{\partial{\xi}} + 
\nonumber\\
( \frac{\partial{\eta}}{\partial{x}} \mathbf{A}_x + 
\frac{\partial{\eta}}{\partial{y}} \mathbf{A}_y + 
\frac{\partial{\eta}}{\partial{z}} \mathbf{A}_z ) \frac{\partial\vec{u}}{\partial{\eta}} + 
\nonumber\\
( \frac{\partial{\zeta}}{\partial{x}} \mathbf{A}_x  + 
\frac{\partial{\zeta}}{\partial{y}} \mathbf{A}_y + 
\frac{\partial{\zeta}}{\partial{z}} \mathbf{A}_z ) \frac{\partial\vec{u}}{\partial{\zeta}}.
\end{eqnarray}

Таким образом, уравнение \ref{matrix_equation} после замены базиса имеет вид:
\begin{equation}
\label{matrix_equation_generalized}
\frac{\partial\vec{u}}{\partial{t}} + 
( \xi_x \mathbf{A}_x  + \xi_y \mathbf{A}_y + \xi_z \mathbf{A}_z ) \frac{\partial\vec{u}}{\partial{\xi}} + 
( \eta_x \mathbf{A}_x + \eta_y \mathbf{A}_y + \eta_z \mathbf{A}_z ) \frac{\partial\vec{u}}{\partial{\eta}} + 
( \zeta_x \mathbf{A}_x  + \zeta_y \mathbf{A}_y + \zeta_z \mathbf{A}_z ) \frac{\partial\vec{u}}{\partial{\zeta}} = \vec f.
\end{equation}

Или, после замены обозначений:
\begin{equation}
\label{matrix_equation_generalized_short}
\frac{\partial\vec{u}}{\partial{t}} + \mathbf{A}_\xi \frac{\partial\vec{u}}{\partial{\xi}} + 
\mathbf{A}_\eta \frac{\partial\vec{u}}{\partial{\eta}} + \mathbf{A}_\zeta \frac{\partial\vec{u}}{\partial{\zeta}} = \vec f.
\end{equation}

Из вида уравнений \ref{matrix_equation_generalized} и \ref{matrix_equation_generalized_short} очевидным образом возникает задача исследования матрицы общего вида:
\begin{equation}
\label{matrix_generalized}
\mathbf{A}_q = q_x \mathbf{A}_x  + q_y \mathbf{A}_y + q_z \mathbf{A}_z,
\end{equation}
где $\mathbf{A}_q$ -- обобщенный вид матрицы, возникающей при переходе в новую систему координат, $q_x$ -- производные базисных векторов нового базиса по старому базису. В случае, когда оба базиса ортонормированные,
\begin{equation}
\label{good_basis_condition}
q_x^2 + q_y^2 + q_z^2 = 1.
\end{equation}
Уравнения \ref{matrix_equation_generalized}, \ref{matrix_equation_generalized_short} и \ref{matrix_generalized} верны и в случае произвольных базисов, в том числе неортогональных, но соотношение \ref{good_basis_condition} в этом случае уже не выполняется.

\subsubsection{Исследование матрицы $\mathbf{A}_q$ в случае линейной упругости}

Рассмотрим частный, но важный случай линейно упругого тела. Он представляет отдельный интерес, так как позволяет получить все необходимые соотношения в аналитическом виде явным образом, и таким образом исследовать, как отражается смена базиса на полной схеме построения численного метода. Кроме того, для большинства рассмотренных моделей вид матриц $\mathbf A_{x_i}$ совпадает с их видом для линейно упругого тела.

В случае линейной упругости в обобщенную матрицу \ref{matrix_generalized} входят матрицы $\mathbf{A}_x, \mathbf{A}_y, \mathbf{A}_z$, явный вид которых приведен в разделе \ref{elastic_matrixes}. Получаем следующий вид обобщенной матрицы (с точностью до знака):

\begin{equation}
\label{matrix_generalized_elastic}
- \mathbf{A}_q =
\left( \begin{array}{cccccccccccc}
0 & 0 & 0 & \frac 1 \rho q_x & \frac 1 \rho q_y & \frac 1 \rho q_z & 0 & 0 & 0 \\ 
0 & 0 & 0 & 0 & \frac 1 \rho q_x & 0 & \frac 1 \rho q_y & \frac 1 \rho q_z & 0 \\ 
0 & 0 & 0 & 0 & 0 & \frac 1 \rho q_x & 0 & \frac 1 \rho q_y & \frac 1 \rho q_z \\ 
(\lambda+2\mu) q_x & \lambda q_y & \lambda q_z & 0 & 0 & 0 & 0 & 0 & 0 \\ 
\mu q_y & \mu q_x & 0 & 0 & 0 & 0 & 0 & 0 & 0 \\ 
\mu q_z & 0 & \mu q_x & 0 & 0 & 0 & 0 & 0 & 0 \\ 
\lambda q_x & (\lambda+2\mu) q_y & \lambda q_z & 0 & 0 & 0 & 0 & 0 & 0 \\ 
0 & \mu q_z & \mu q_y & 0 & 0 & 0 & 0 & 0 & 0 \\ 
\lambda q_x & \lambda q_y & (\lambda+2\mu) q_z & 0 & 0 & 0 & 0 & 0 & 0  
\end{array} \right)
\end{equation}

Полученная матрица имеет следующую структуру:

\begin{eqnarray}
- \mathbf{A}_q =
\left( \begin{array}{cccccccccccc}
0 & \mathbf{B} \\
\mathbf{C} & 0
\end{array} \right),
\nonumber\\
\mathbf{B} = 
\left( \begin{array}{cccccccccccc}
\frac 1 \rho q_x & \frac 1 \rho q_y & \frac 1 \rho q_z & 0 & 0 & 0 \\ 
0 & \frac 1 \rho q_x & 0 & \frac 1 \rho q_y & \frac 1 \rho q_z & 0 \\ 
0 & 0 & \frac 1 \rho q_x & 0 & \frac 1 \rho q_y & \frac 1 \rho q_z 
\end{array} \right),
\nonumber\\
\mathbf{C} = 
\left( \begin{array}{cccccccccccc}
(\lambda+2\mu) q_x & \lambda q_y & \lambda q_z \\ 
\mu q_y & \mu q_x & 0 \\ 
\mu q_z & 0 & \mu q_x \\ 
\lambda q_x & (\lambda+2\mu) q_y & \lambda q_z \\ 
0 & \mu q_z & \mu q_y \\ 
\lambda q_x & \lambda q_y & (\lambda+2\mu) q_z  
\end{array} \right)
\end{eqnarray}

При нахождении собственных значений и векторов по методу Челнокова \cite{chelnokov} задача сводится к нахождению их для матрицы $\mathbf{C}^T \mathbf{B}^T$, которая имеет следующий вид:

\begin{eqnarray}
\frac 1 \rho 
\left( \begin{array}{cccccccccccc}
(\lambda+2\mu) q_x^2 + \mu q_y^2 + \mu q_z^2 & (\lambda+\mu) q_x q_y & (\lambda+\mu) q_x q_z \\
(\lambda+\mu) q_x q_y & \mu q_x^2 + (\lambda+2\mu) q_y^2 + \mu q_z^2 & (\lambda+\mu) q_y q_z \\
(\lambda+\mu) q_x q_z & (\lambda+\mu) q_y q_z & \mu q_x^2 + \mu q_y^2 + (\lambda+2\mu) q_z^2  
\end{array} \right).
\end{eqnarray} 

Собственные числа имеют вид:
\begin{eqnarray}
\left( \begin{array}{cccccccccccc}
\lambda_1 \\
\lambda_2 \\
\lambda_3 
\end{array} \right) = 
\frac 1 \rho (q_x^2 + q_y^2 + q_z^2)
\left( \begin{array}{cccccccccccc}
(\lambda+2\mu) \\
\mu \\
\mu  
\end{array} \right).
\end{eqnarray} 

Таким образом, при смене базиса собственные числа $\lambda_i$ меняются в $q_x^2 + q_y^2 + q_z^2$ раз. Если оба базиса ортонормированные, то соотношение \ref{good_basis_condition} выполняется и собственные числа не меняются при смене базиса. В случае же перехода в неортонормированный базис собственные числа могут меняться в широком диапазоне.

От $\lambda_i$ напрямую зависит, какие точки на предыдущем шаге по времени будут нужны для реконструкции решения на новом временном слое. Чем больше значения $\lambda_i$, тем больше угол наклона характеристики (см.ниже \todo{ссылка}). При увеличении $\lambda_i$ следует ожидать уменьшения допустимого шага по времени (при использовании курантовского ограничения на шаг $\lambda\tau / h \le 1$), либо (при попытке сохранить шаг по времени неизменным) еще более неприятных последствий, таких как появление непредусмотренных характеристик, выводящих за пределы расчетной области. Действительно, экспериментально было получено, что при попытке использовать неортогональный базис для расчета граничных узлов происходит увеличение $\lambda_i$ при переходе в новый базис и проявляются оба эти эффекта.
