\section{Математическая модель}

\subsection{Уравнения механики твёрдого тела}
Для математического моделирования волновых процессов в деформируемом твёрдом
теле используется система динамических уравнений \cite{novatsky,sedov} в виде
\begin{eqnarray}
\label{initial_equations}
\rho\dot{v}_i=\nabla_j\sigma_{ij}+f_i & \textrm{(уравнения движения)}\nonumber\\
\sigma_{ij}=q_{ijkl}\dot{\varepsilon}_{kl}+F_{ij} & \textrm{(реологические
соотношения).}
\end{eqnarray}

Здесь $\rho$ – плотность среды, $v_i$ – компоненты скорости смещения,
$\sigma_{ij}$, $\varepsilon_{ij}$ -- компоненты тензоров напряжений и деформаций,
$\nabla_j$ – ковариантная производная по $j$-й координате, $f_i$ – массовые
силы, действующие на единицу объёма, $F_{ij}$ -- правая часть, используемая, например, для описания диссипации в моделях с учётом вязкости.

В случае малых деформаций тензор скоростей деформаций $e_{ij}=\dot{\varepsilon}_{ij}$ 
выражается через компоненты скорости смещения линейным образом:
\begin{equation}
e_{ij}=\frac{1}{2}(\nabla_j v_i+\nabla_i v_j).
\end{equation}

Вид компонент тензора 4-го порядка $q_{ijkl}$ и правой части $F_{ij}$ определяется реологией среды.

Для замыкания системы уравнений \ref{initial_equations} её необходимо дополнить
уравнением состояния, определяющим зависимость плотности от напряжений:
$$\rho=\rho_0e^{\frac{p}{K}},$$
где $p=-\frac{1}{3}\sum\sigma_{kk}$ -- давление, $K=\lambda+\frac{2}{3}\mu$ --
коэффициент всестороннего сжатия, $\lambda$ и $\mu$ -- параметры Ламе.

Параметры Ламе зависят от материала и связаны с модулем продольной упругости и коэффициентом Пуассона следующим образом:
\begin{eqnarray}
\label{lame_parameters}
\lambda=\frac{E\nu}{(1+\nu)(1-2\nu)}
\nonumber\\
\mu=G=\frac{E}{2(1+\nu)}
\end{eqnarray}
Здесь $E$ -- модуль продольной упругости, $\nu$ -- коэффициент Пуассона, $G$ -- модуль сдвига.

В простейшем случае линейной упругости $q_{ijkl}=\lambda\delta_{ij}\delta_{kl}+\mu(\delta_{ik}\delta_{jl}+\delta_{il}\delta_{jk})$ и $F_{ij}=0$. Тогда в приближении малых деформаций и в отсутствии внешних сил в трехмерном пространстве и декартовых координатах уравнения \ref{initial_equations} принимают вид

\begin{eqnarray}
\label{simple_equations}
\frac{\partial{v_x}}{\partial{t}}=\frac{1}{\rho}(\frac{\partial{\sigma_{xx}}}{\partial{x}}+\frac{\partial{\sigma_{xy}}}{\partial{y}}+\frac{\partial{\sigma_{xz}}}{\partial{z}})
\nonumber\\
\frac{\partial{v_y}}{\partial{t}}=\frac{1}{\rho}(\frac{\partial{\sigma_{xy}}}{\partial{x}}+\frac{\partial{\sigma_{yy}}}{\partial{y}}+\frac{\partial{\sigma_{yz}}}{\partial{z}})
\nonumber\\
\frac{\partial{v_z}}{\partial{t}}=\frac{1}{\rho}(\frac{\partial{\sigma_{xz}}}{\partial{x}}+\frac{\partial{\sigma_{yz}}}{\partial{y}}+\frac{\partial{\sigma_{zz}}}{\partial{z}})
\nonumber\\
\frac{\partial{\sigma_{xx}}}{\partial{t}}=(\lambda+2\mu)\frac{\partial{v_x}}{\partial{x}}+\lambda\frac{\partial{v_y}}{\partial{y}}+\lambda\frac{\partial{v_z}}{\partial{z}}
\nonumber\\
\frac{\partial{\sigma_{xy}}}{\partial{t}}=\mu(\frac{\partial{v_x}}{\partial{y}}+\frac{\partial{v_y}}{\partial{x}})
\nonumber\\
\frac{\partial{\sigma_{xz}}}{\partial{t}}=\mu(\frac{\partial{v_x}}{\partial{z}}+\frac{\partial{v_z}}{\partial{x}})
\nonumber\\
\frac{\partial{\sigma_{yy}}}{\partial{t}}=\lambda\frac{\partial{v_x}}{\partial{x}}+(\lambda+2\mu)\frac{\partial{v_y}}{\partial{y}}+\lambda\frac{\partial{v_z}}{\partial{z}}
\nonumber\\
\frac{\partial{\sigma_{yz}}}{\partial{t}}=\mu(\frac{\partial{v_z}}{\partial{y}}+\frac{\partial{v_y}}{\partial{z}})
\nonumber\\
\frac{\partial{\sigma_{zz}}}{\partial{t}}=\lambda\frac{\partial{v_x}}{\partial{x}}+\lambda\frac{\partial{v_y}}{\partial{y}}+(\lambda+2\mu)\frac{\partial{v_z}}{\partial{z}}
\end{eqnarray}

Очевидно, что уравнения \ref{simple_equations} можно переписать в матричной форме:
\begin{equation}
\label{simple_matrix_equation}
\frac{\partial\vec{u}}{\partial{t}}+\mathbf{A}_x\frac{\partial\vec{u}}{\partial{x}}+
\mathbf{A}_y\frac{\partial\vec{u}}{\partial{y}}+
\mathbf{A}_z\frac{\partial\vec{u}}{\partial{z}}=0.
\end{equation}
Здесь
$\vec{u}=\{v_x,v_y,v_z,\sigma_{xx},\sigma_{yy},\sigma_{zz},\sigma_{xy},\sigma_{xz},\sigma_{yz}\}^T$
-- вектор искомых функций, $x,y,z$ --  независимые пространственные переменные, $t$ -- время.

Аналогично можно записать более общую систему \ref{initial_equations} в виде:
\begin{equation}
\label{matrix_equation}
\frac{\partial\vec{u}}{\partial{t}}+\mathbf{A}_x\frac{\partial\vec{u}}{\partial{x}}+
\mathbf{A}_y\frac{\partial\vec{u}}{\partial{y}}+
\mathbf{A}_z\frac{\partial\vec{u}}{\partial{z}}=\vec{f}.
\end{equation}
Здесь $\vec{f}$ -- вектор правых частей, размерность которого равна размерности исходной системы, а выражения для компонентов зависят от реологии среды. Точный вид матриц $\mathbf{A}_x$, $\mathbf{A}_y$, $\mathbf{A}_z$ также зависит от реологии среды.

\clearpage
\newpage

\subsection{Приближение линейно упругого тела}

\subsubsection{Реологические соотношения для линейно упругого тела}

Для линейно упругого тела тензор $q_{ijkl}$ и правая часть $F_{ij}$ в \ref{initial_equations} принимают следующий вид:
\begin{eqnarray}
\label{tensor_qijkl_elastic}
q_{ijkl}=\lambda\delta_{ij}\delta_{kl}+\mu(\delta_{ik}\delta_{jl}+\delta_{il}
\delta_{jk}),\nonumber\\
F_{ij}=0.
\end{eqnarray}
В этом соотношении $\lambda$ и $\mu$ -- параметры Ламе, $\delta_{ij}$ -- символ Кронекера.

\subsubsection{Матричная форма уравнений в случае линейно упругого тела}

Для линейно упругого тела матрицы $\mathbf{A}_x$, $\mathbf{A}_y$, $\mathbf{A}_z$ в \ref{matrix_equation} принимают следующий вид.

\begin{displaymath}
\mathbf{A}_x =
\left( \begin{array}{cccccccccccc}
0 & 0 & 0 & -\frac 1 \rho & 0 & 0 & 0 & 0 & 0 \\ 
0 & 0 & 0 & 0 & -\frac 1 \rho & 0 & 0 & 0 & 0 \\ 
0 & 0 & 0 & 0 & 0 & -\frac 1 \rho & 0 & 0 & 0 \\ 
-(\lambda+2\mu) & 0 & 0 & 0 & 0 & 0 & 0 & 0 & 0 \\ 
0 & -\mu & 0 & 0 & 0 & 0 & 0 & 0 & 0 \\ 
0 & 0 & -\mu & 0 & 0 & 0 & 0 & 0 & 0 \\ 
-\lambda & 0 & 0 & 0 & 0 & 0 & 0 & 0 & 0 \\ 
0 & 0 & 0 & 0 & 0 & 0 & 0 & 0 & 0 \\ 
-\lambda & 0 & 0 & 0 & 0 & 0 & 0 & 0 & 0  
\end{array} \right),
\end{displaymath} 
\begin{displaymath}
\mathbf{A}_y =
\left( \begin{array}{cccccccccccc}
0 & 0 & 0 & 0 & -\frac 1 \rho & 0 & 0 & 0 & 0 \\ 
0 & 0 & 0 & 0 & 0 & 0 & -\frac 1 \rho & 0 & 0 \\ 
0 & 0 & 0 & 0 & 0 & 0 & 0 & -\frac 1 \rho & 0 \\ 
0 & -\lambda & 0 & 0 & 0 & 0 & 0 & 0 & 0 \\ 
-\mu & 0 & 0 & 0 & 0 & 0 & 0 & 0 & 0 \\ 
0 & 0 & 0 & 0 & 0 & 0 & 0 & 0 & 0 \\ 
0 & -(\lambda+2\mu) & 0 & 0 & 0 & 0 & 0 & 0 & 0 \\ 
0 & 0 & -\mu & 0 & 0 & 0 & 0 & 0 & 0 \\ 
0 & -\lambda & 0 & 0 & 0 & 0 & 0 & 0 & 0  
\end{array} \right),
\end{displaymath}
\begin{displaymath}
\mathbf{A}_z =
\left( \begin{array}{cccccccccccc}
0 & 0 & 0 & 0 & 0 & -\frac 1 \rho & 0 & 0 & 0 \\ 
0 & 0 & 0 & 0 & 0 & 0 & 0 & -\frac 1 \rho & 0 \\ 
0 & 0 & 0 & 0 & 0 & 0 & 0 & 0 & -\frac 1 \rho \\ 
0 & 0 & -\lambda & 0 & 0 & 0 & 0 & 0 & 0 \\ 
0 & 0 & 0 & 0 & 0 & 0 & 0 & 0 & 0 \\ 
-\mu & 0 & 0 & 0 & 0 & 0 & 0 & 0 & 0 \\ 
0 & 0 & -\lambda & 0 & 0 & 0 & 0 & 0 & 0 \\ 
0 & -\mu & 0 & 0 & 0 & 0 & 0 & 0 & 0 \\ 
0 & 0 & -(\lambda+2\mu) & 0 & 0 & 0 & 0 & 0 & 0  
\end{array} \right).
\end{displaymath}

\clearpage
\newpage

\subsection{Приближение упруго-пластического тела}

\subsubsection{Реологические соотношения для упруго-пластического тела}

Для упруго-пластического тела тензор $q_{ijkl}$ и правая часть $F_{ij}$ в \ref{initial_equations} имеют более сложный вид:
\begin{eqnarray}
\label{tensor_qijkl_plastic}
q_{ijkl}=\lambda\delta_{ij}\delta_{kl}+\mu(\delta_{ik}\delta_{jl}+\delta_{il}\delta_{jk})-\frac{I\mu\sigma_{ij}\sigma_{kl}}{K^2},
\nonumber\\
F_{ij}=0.
\end{eqnarray}
В этом соотношении $\lambda$ и $\mu$ -- параметры Ламе, $K$ -- предел текучести на сдвиг, $\sigma_{ij}$ -- компоненты тензора напряжений, $\delta_{ij}$ -- символ Кронекера, $I$ -- параметр модели, который определяется следующим образом:
\todo{Имени кого модель? И точный вид S для 3D?}
\begin{equation}
\label{I_parameter_plastic}
I=\begin{cases}
0, & \text{если $S = \sigma_{xx}^2+\sigma_{yy}^2+\sigma_{zz}^2+2\sigma_{xy}^2 < 2K^2$}\\
1, & \text{если $S >= 2K^2$}.
\end{cases}
\end{equation}

\subsubsection{Матричная форма уравнений в случае упруго-пластического тела}

Для линейно упругого тела матрицы $\mathbf{A}_x$, $\mathbf{A}_y$, $\mathbf{A}_z$ в \ref{matrix_equation} имеют существенно более сложный вид, так как компоненты тензора $q_{ijkl}$ зависят от компонентов тензора $\sigma$. Значения $\sigma_{ij}$ в общем случае различны в каждой точке пространства в каждый момент времени. Это приводит к тому, что невозможно упростить вид матриц аналитически и получить их покомпонентную запись в терминах $\lambda, \mu, \rho$, как это было сделано для линейно упругого тела. Значения ненулевых элементов каждой матрицы необходимо вычислять в каждой точке пространства в каждый момент времени в соответствии с \ref{tensor_qijkl_plastic} и \ref{I_parameter_plastic}, используя текущие значения $\sigma_{ij}$ в данной точке.

\begin{displaymath}
\mathbf{A}_x =
\left( \begin{array}{cccccccccccc}
0 & 0 & 0 & -\frac 1 \rho & 0 & 0 & 0 & 0 & 0 \\ 
0 & 0 & 0 & 0 & -\frac 1 \rho & 0 & 0 & 0 & 0 \\ 
0 & 0 & 0 & 0 & 0 & -\frac 1 \rho & 0 & 0 & 0 \\ 
-q_{1111} & -\frac{q_{1112}+q_{1121}}{2} & -\frac{q_{1113}+q_{1131}}{2} & 0 & 0 & 0 & 0 & 0 & 0 \\ 
-q_{1211} & -\frac{q_{1212}+q_{1221}}{2} & -\frac{q_{1213}+q_{1231}}{2} & 0 & 0 & 0 & 0 & 0 & 0 \\ 
-q_{1311} & -\frac{q_{1312}+q_{1321}}{2} & -\frac{q_{1313}+q_{1331}}{2} & 0 & 0 & 0 & 0 & 0 & 0 \\ 
-q_{2211} & -\frac{q_{2212}+q_{2221}}{2} & -\frac{q_{2213}+q_{2231}}{2} & 0 & 0 & 0 & 0 & 0 & 0 \\ 
-q_{2311} & -\frac{q_{2312}+q_{2321}}{2} & -\frac{q_{2313}+q_{2331}}{2} & 0 & 0 & 0 & 0 & 0 & 0 \\ 
-q_{3311} & -\frac{q_{3312}+q_{3321}}{2} & -\frac{q_{3313}+q_{3331}}{2} & 0 & 0 & 0 & 0 & 0 & 0  
\end{array} \right),
\end{displaymath} 
\begin{displaymath}
\mathbf{A}_y =
\left( \begin{array}{cccccccccccc}
0 & 0 & 0 & 0 & -\frac 1 \rho & 0 & 0 & 0 & 0 \\ 
0 & 0 & 0 & 0 & 0 & 0 & -\frac 1 \rho & 0 & 0 \\ 
0 & 0 & 0 & 0 & 0 & 0 & 0 & -\frac 1 \rho & 0 \\ 
-\frac{q_{1112}+q_{1121}}{2} & -q_{1122} & -\frac{q_{1123}+q_{1132}}{2} & 0 & 0 & 0 & 0 & 0 & 0 \\ 
-\frac{q_{1212}+q_{1221}}{2} & -q_{1222} & -\frac{q_{1223}+q_{1232}}{2} & 0 & 0 & 0 & 0 & 0 & 0 \\ 
-\frac{q_{1312}+q_{1321}}{2} & -q_{1322} & -\frac{q_{1323}+q_{1332}}{2} & 0 & 0 & 0 & 0 & 0 & 0 \\ 
-\frac{q_{2212}+q_{2221}}{2} & -q_{2222} & -\frac{q_{2223}+q_{2232}}{2} & 0 & 0 & 0 & 0 & 0 & 0 \\ 
-\frac{q_{2312}+q_{2321}}{2} & -q_{2322} & -\frac{q_{2323}+q_{2332}}{2} & 0 & 0 & 0 & 0 & 0 & 0 \\ 
-\frac{q_{3312}+q_{3321}}{2} & -q_{3322} & -\frac{q_{3323}+q_{3332}}{2} & 0 & 0 & 0 & 0 & 0 & 0  
\end{array} \right),
\end{displaymath}
\begin{displaymath}
\mathbf{A}_z =
\left( \begin{array}{cccccccccccc}
0 & 0 & 0 & 0 & 0 & -\frac 1 \rho & 0 & 0 & 0 \\ 
0 & 0 & 0 & 0 & 0 & 0 & 0 & -\frac 1 \rho & 0 \\ 
0 & 0 & 0 & 0 & 0 & 0 & 0 & 0 & -\frac 1 \rho \\ 
-\frac{q_{1113}+q_{1131}}{2} & -\frac{q_{1123}+q_{1132}}{2} & -q_{1133} & 0 & 0 & 0 & 0 & 0 & 0 \\ 
-\frac{q_{1213}+q_{1231}}{2} & -\frac{q_{1223}+q_{1232}}{2} & -q_{1233} & 0 & 0 & 0 & 0 & 0 & 0 \\ 
-\frac{q_{1313}+q_{1331}}{2} & -\frac{q_{1323}+q_{1332}}{2} & -q_{1333} & 0 & 0 & 0 & 0 & 0 & 0 \\ 
-\frac{q_{2213}+q_{2231}}{2} & -\frac{q_{2223}+q_{2232}}{2} & -q_{2233} & 0 & 0 & 0 & 0 & 0 & 0 \\ 
-\frac{q_{2313}+q_{2331}}{2} & -\frac{q_{2323}+q_{2332}}{2} & -q_{2333} & 0 & 0 & 0 & 0 & 0 & 0 \\ 
-\frac{q_{3313}+q_{3331}}{2} & -\frac{q_{3323}+q_{3332}}{2} & -q_{3333} & 0 & 0 & 0 & 0 & 0 & 0  
\end{array} \right).
\end{displaymath}

\clearpage
\newpage

\subsection{Преобразование уравнений при смене базиса}

\todo{Вывод. Вид матриц. Анализ матриц. Неортогональный базис. Исследование последствий для шага по времени.}
