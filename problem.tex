\section{Постановка задачи}

Рассматривается задача о динамическом нагружении элемента композитной обшивки крыла самолёта. На рис.
\ref{pic:construction_intro} приведена схема строения обшивки и силового кессона крыла. Обшивка толщиной 6.5~мм состоит из 3 композитных субпакетов, стрингер толщиной 13~мм -- из 6 аналогичных субпакетов.

\begin{figure}[h]
\center{\includegraphics[width=\textwidth]{png/construction.png}}
\caption{Обшивка и силовой кессон крыла.}
\label{pic:construction_intro}
\end{figure}

В эксперименте по непробивающему воздействию на обшивку нагрузка создается стальным 
ударником цилиндрической формы с диаметром закругления на конце 25.4~мм.

Каждый субпакет состоит из 11 монослоёв со взаимной ориентацией при укладке 
45/0/-45/0/0/90/0/0/-45/0/45. Упругие характеристики монослоёв приведены в табл. \ref{tbl:subpackage_intro}.

\begin{table}[h]
\centering
\caption{Упругие характеристики монослоёв}
\begin{tabular}{|p{3cm}|c|c|c|c|c|c|c|}
\hline
Элемент & E, ГПа & $\nu$ & $\rho$, кг/м$^{3}$ & $\lambda$, ГПа & $\mu$, ГПа &
$c_p$, м/с & $c_s$, м/с \\
\hline
Монослой & 8.5 & 0.32 & 1580 & 5.72 & 3.22 & 2775 & 1425 \\
\hline
\end{tabular}
\label{tbl:subpackage_intro}
\end{table}

Рассматриваются две постановки -- удар по отдельному элементу обшивки и удар по элементу обшивки со стрингером.

Задачей численного моделирования является получение волновых процессов в конструкции при динамическом воздействии с учетом взаимодействия волновых фронтов, влияния внешних и внутренних контактных границ. В результате должны быть получены области максимальных напряжений и потенциальных разрушений в образце. Прочностные характеристики монослоёв приведены в таблице \ref{tbl:max_stresses_intro}.

\begin{table}[h]
\centering
\caption{Прочностные характеристики монослоёв}
\begin{tabular}{|l|c|}
\hline
Тип нагрузки & Предельно допустимая нагрузка, МПа \\
\hline
Растяжение вдоль волокон & 2630 \\
Сжатие вдоль волокон & 1530 \\
Растяжение поперёк волокон & 86 \\
Сжатие поперёк волокон & 213 \\
Сдвиг & 112 \\
\hline
\end{tabular}
\label{tbl:max_stresses_intro}
\end{table}

